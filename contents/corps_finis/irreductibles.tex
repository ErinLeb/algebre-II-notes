\subsection{Polynômes irreductibles sur un corps fini}


Soit $I(n,q)$ le plus petit polynôme unitaire de degré $n$ sur $\F_q$.

\begin{theorem}
	Pour tout $q$ puissance d'un nombre premier et $n \geq 1$ pm a $I(n,q) > 0$.\\

	Plus précisément:
	$$ I(n,q) = \frac{1}{n} \sum_{d|n} \mu(\frac{n}{d}) q^d$$

	où $\mu$ est la fonction de Möbius:

	\[ \mu(k) = \left\{\begin{array}{ll}
			0      & \text{si il existe } l \text{ premier tel que } l^2 \text{ divise } k                            \\
			(-1)^r & \text{si } k = p_1 \cdot p_2 \cdot \ldots \cdot p_r \text{ avec } p_i \text{ premiers distincts}
		\end{array} \right.\]

\end{theorem}

\begin{proof}
	%TODO
\end{proof}

\begin{example}
	$ q = 2, \ n = 4, \ 2^4 = 16 $
	$$	X ^16 - X = X(X-1)(X^2 + X + 1)(X^4 + X + 1)(X^4 + X^3 + 1)(X^4 + x^3 + x^2 + x + 1) $$
	$$	I(2,2) = \frac{1}{2}\sum_{d|2} \mu(\frac{2}{d}) 2^d = \frac{1}{2}(4-2) = 1$$
	$$	I(4,2) = \frac{1}{4}\sum_{d|4} \mu(\frac{4}{d}) 2^d = \frac{1}{4}(2^4 - 2^2) = 3$$
\end{example}


\begin{coro}
	Pour tout $n$ et tout $ $ puissance d'un nombre premier:
	$$ I (n, q) > 0 $$
\end{coro}


\begin{proof}
	\begin{eqnarray*}
		n I(n, q) & = & \sum_{d|n} \mu(\frac{n}{d}) q^d \\
		&>& q^n - \sum_{d|n, d < n} q^d \\
		&>& q^n - \sum_{1 \leq d\leq \frac{n}{2}} q^d \\
		&=& q^n - q\frac{q^{\frac{n}{2}} - 1}{q-1} \\ %TODO: Use partie entière
		&>& 0
	\end{eqnarray*}
\end{proof}


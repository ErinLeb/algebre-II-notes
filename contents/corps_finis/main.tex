
\section{Corps finis}

Un corps fini est un corps qui a un nombre fini d'éléments.

Sa caractéristique est forcement $p$ premier.
Si c'était 0 alors $\Z$ s'injecterait dans $K$ et $\Z$ est infini.

$K$ peut etre vu comme une extension de $\F_p$ via le morphisme:
\begin{eqnarray*}
	\F_p & \to & K \\
	\bar{1} &\mapsto& 1_K
\end{eqnarray*}

$K$ est en particulier un space vectoriel de dimension finie ${K:\F_p}$ et $|K| = p ^{[K:\F_P]} $


Ici $\F_p = \left( \Zn{p}, +, \times \right)$, $p$ premier.
A isomorphisme près il y a un seul corps a $p$ éléments.

$\F_p = \left( \Zn, +, \times \right)$ est un corps si et seulement si $n$ est premier.

\begin{theorem}
	Soit $q = p^n$, avec $p$ premier et $n \geq 1$. Alors il existe un corps de cardinal $q$ unique a isomorphisme près.
	C'est le corps de décomposition por $X^q-X$ sur $\F_p$. On le note $\F_q$.
\end{theorem}

\begin{proof}
	Soit $K$ le corps de décomposition de $X^q-C$ sur $F_p$. \\
	On note $K'$ l'ensemble des racines das $K$ de $X^q-X$. \\
	$K'$ est au fait un corps: $0, 1 \in K'$. Si $x, y \in K'$'.\\
	Montrons que $(x+y)^q = x^q + y^q = x+ y$.\\
	Pour montrer l'identité precedente, on observe que pour tout $>0$ et pour tout $x,y \in K$ on a
	$(x+y)^{p^k}= x^{p^k} + y^{p^k}$. Cela se montre par récurrence grâce a la formule $(x+y)^p= x^p + y^p$.\\
	$p \mid \binom{p}{k}$ lorsque $1 \leq k \leq p-1$ aussi $(x+y)^p = \sum^p_{k=0} \binom{p}{k} x^k y^{p-k} = x^p + y^p$\\

	$x\in K'$, alors $-x$ et aussi dans $K'$.
	En effet si $car K = 2$, $-x = x -2x = x$. \\
	Si $car K$ est impair $(-x)^q = -x^q = -x$. \\
	Évidement $x,y \in K' \implies xy \in K'$, car $(x*y)^q = x^qy^q = xy$\\
	$x\in K' x \neq 0 \implies \frac{1}{x} \in K'$.

	Alors $'$ est un sous corps de $K$.\\
	Donc par la définition de corps de décomposition $K' = K$.\\

	La dérivée de $X^q-X$ est $qX^{q-1} - X$ et $qX^{q-1} - 1 = 1$\\
	Donc toutes les racines du polynôme sont simples et il y a exactement $q$. $K$ est bien un corps de cardinal $q$.\\

	%TODO: Add rappel


	Après l'existence, montrons l'unicité.\\

	On considère $L$ un corps de cardinal $q$.\\
	On sait que $\forall x \in L \setminus \{0\},\  x^{q-1} = 1$ (théorème de Lagrange).\\ %TODO: Explanation
	$\forall x \in L, \ x^q -x = 0$.\\
	$X^q-X$ est scindé dans $L$ (car il a $q$ racines distinctes).\\ %TODO: Add explanation
	$L$ contient un corps de décomposition pour $X^q-X$ sur $\F_p$. Autrement dit un $K_1$ qui est
	isomorphe à $K$ (d'après les propriétés de décomposition).\\
	$L_1 \subset L$ et $|K_1| = q = |L| \implies K_1 = L \cong K$.
\end{proof}

\begin{example}

	\begin{itemize}
		\item $X^2+X+1$ est irréductible sur $\F_2$ et
		      $$\F_2[X]/ <x^2+x+1> \cong F_4$$

		      On prend $\alpha$ une racine de $x^2+x+1$.
		      $$\F_2[X]/ <x^2+x+1> = \set{ a + \alpha b \mid a,b \in \F_2}$$
		      La table de multiplication:

		      \begin{center} %TODO: Add explanation
			      \begin{tabular}{c|c|c|c|c}
				      $*$        & 0          & 1        & $\alpha$    & $1+\alpha$  \\
				      \hline
				      0          & 0          & 1        & $\alpha$    & $1+\alpha$  \\
				      \hline
				      1          & 1          & 1        & $\alpha$    & $\alpha +1$ \\
				      \hline
				      $\alpha$   & $\alpha$   & $\alpha$ & $\alpha +1$ & 1           \\
				      \hline
				      $1+\alpha$ & $1+\alpha$ & $\alpha$ & 1           & $\alpha$    \\
			      \end{tabular}
		      \end{center}
		\item$x^3+x+1$ est irréductible sur $\F_2$ et $\F_2[X]/ <x^3+x+1> \cong F_6$.
		\item$ \F_3[X]/ <x^2+1> \cong F_9$
	\end{itemize}
\end{example}


\begin{exercice}
	Si $K$ est un corps fini et $P \in K[X]$ irréductible sur $K$, alors le
	corps de rupture de $P$ sur $K$ est aussi un corps de décomposition pour $P$ sur $K$.
\end{exercice}

\begin{remarque}
	$\F_{p^n}$ est une extension de $\F_{p^m}$ \ssi $m$ divise $n$.\\
	Ainsi $\F_8$ n'est pas une extension de $\F_4$. \\
    %TODO: Add proof
\end{remarque}

%ODO: Add Frobenius


\subsection{Polynômes irreductibles sur un corps fini}


Soit $I(n,q)$ le plus petit polynôme unitaire de degré $n$ sur $\F_q$.

\begin{theorem}
	Pour tout w$q$ puissance d'un nombre premier et $n \geq 1$ pm a $I(n,q) > 0$.\\

	Plus précisément:
	$$ I(n,q) = \frac{1}{n} \sum_{d|n} \mu(\frac{n}{d}) q^d$$

	où $\mu$ est la fonction de Möbius:

	\[ \mu(k) = \begin{array}{ll}
			0      & \text{si il existe } l \text{ premier tel que } l^2 \text{ divise } k                            \\
			(-1)^r & \text{si } k = p_1 \cdot p_2 \cdot \ldots \cdot p_r \text{ avec } p_i \text{ premiers distincts}
		\end{array} \]

\end{theorem}

\begin{proof}
	%TODO
\end{proof}



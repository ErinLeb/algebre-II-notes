
\section{Corps finis}

Un corps fini est un corps qui a un nombre fini d'éléments.

Sa caractéristique est forcément $p$ premier.
Si c'était 0, alors $\Z$ s'injecterait dans $K$ et $\Z$ serait infini.

$K$ peut être vu comme une extension de $\F_p$ via le morphisme:
\begin{eqnarray*}
	\F_p & \to & K \\
	\bar{1} &\mapsto& 1_K
\end{eqnarray*}

$K$ est en particulier un espace vectoriel de dimension finie ${K:\F_p}$ et $|K| = p ^{[K:\F_P]} $


Ici $\F_p = \left( \Zn{p}, +, \times \right)$, $p$ premier.
A isomorphisme près, il y a un seul corps à $p$ éléments.

$\F_p = \left( \Zn, +, \times \right)$ est un corps \ssi $n$ est premier.

\begin{theorem}
	Soit $q = p^n$, avec $p$ premier et $n \geq 1$. Alors il existe un corps de cardinal $q$ unique à isomorphisme près.
	C'est le corps de décomposition pour $X^q-X$ sur $\F_p$. On le note $\F_q$.
\end{theorem}

\begin{proof}
	\begin{itemize}
		\item Existence:\\

		      Soit $K$ le corps de décomposition de $X^q-C$ sur $F_p$. \\
		      On note $K'$ l'ensemble des racines dans $K$ de $X^q-X$. \\
		      $K'$ est en fait un corps: $0, 1 \in K'$. Si $x, y \in K'$'.\\
		      Montrons que $(x+y)^q = x^q + y^q = x+ y$.\\
		      Pour montrer l'identité précédente, on observe que pour tout $>0$ et pour tout $x,y \in K$ on a
		      $(x+y)^{p^k}= x^{p^k} + y^{p^k}$. Cela se montre par récurrence grâce à la formule $(x+y)^p= x^p + y^p$.\\
		      $p \mid \binom{p}{k}$ lorsque $1 \leq k \leq p-1$ aussi $(x+y)^p = \sum^p_{k=0} \binom{p}{k} x^k y^{p-k} = x^p + y^p$\\

		      $x\in K'$, alors $-x$ est aussi dans $K'$.
		      En effet, si $car K = 2$, $-x = x -2x = x$. \\
		      Si $car K$ est impaire, $(-x)^q = -x^q = -x$. \\
		      Évidemment $x,y \in K' \implies xy \in K'$, car $(x*y)^q = x^qy^q = xy$\\
		      $x\in K' x \neq 0 \implies \frac{1}{x} \in K'$.

		      Alors $K'$ est un sous corps de $K$.\\
		      Donc par la définition de corps de décomposition $K' = K$.\\

		      La dérivée de $X^q-X$ est $qX^{q-1} - X$ et $qX^{q-1} - 1 = 1$\\
		      Donc toutes les racines du polynôme sont simples et il y en a exactement $q$. $K$ est bien un corps de cardinal $q$.\\

		      %TODO: Add rappel
		\item Unicité:\\

		      On considère $L$ un corps de cardinal $q$.\\
		      On sait que $\forall x \in L \setminus \{0\},\  x^{q-1} = 1$ (théorème de Lagrange).\\ %TODO: Explanation
		      $\forall x \in L, \ x^q -x = 0$.\\
		      $X^q-X$ est scindé dans $L$ (car il a $q$ racines distinctes).\\ %TODO: Add explanation
		      $L$ contient un corps de décomposition pour $X^q-X$ sur $\F_p$. Autrement dit, un $K_1$ qui est
		      isomorphe à $K$ (d'après les propriétés de décomposition).\\
		      $L_1 \subset L$ et $|K_1| = q = |L| \implies K_1 = L \cong K$.
	\end{itemize}
\end{proof}

\begin{example}

	\begin{itemize}
		\item $X^2+X+1$ est irréductible sur $\F_2$ et
		      $$\F_2[X]/ <x^2+x+1> \cong F_4$$

		      On prend $\alpha$ une racine de $x^2+x+1$.
		      $$\F_2[X]/ <x^2+x+1> = \set{ a + \alpha b \mid a,b \in \F_2}$$
		      La table de multiplication:

		      \begin{center} %TODO: Add explanation
			      \begin{tabular}{c|c|c|c|c}
				      $*$        & 0          & 1        & $\alpha$    & $1+\alpha$  \\
				      \hline
				      0          & 0          & 1        & $\alpha$    & $1+\alpha$  \\
				      \hline
				      1          & 1          & 1        & $\alpha$    & $\alpha +1$ \\
				      \hline
				      $\alpha$   & $\alpha$   & $\alpha$ & $\alpha +1$ & 1           \\
				      \hline
				      $1+\alpha$ & $1+\alpha$ & $\alpha$ & 1           & $\alpha$    \\
			      \end{tabular}
		      \end{center}
		\item$x^3+x+1$ est irréductible sur $\F_2$ et $\F_2[X]/ <x^3+x+1> \cong F_6$.
		\item$ \F_3[X]/ <x^2+1> \cong F_9$
	\end{itemize}
\end{example}


\begin{exercice}
	Si $K$ est un corps fini et $P \in K[X]$ irréductible sur $K$, alors le
	corps de rupture de $P$ sur $K$ est aussi un corps de décomposition pour $P$ sur $K$.
\end{exercice}

\begin{remarque}
	$\F_{p^n}$ est une extension de $\F_{p^m}$ \ssi $m$ divise $n$.\\
	Ainsi $\F_8$ n'est pas une extension de $\F_4$. \\
	%TODO: Add proof
\end{remarque}

%TODO: Add Frobenius


\subsection{Polynômes irreductibles sur un corps fini}


Soit $I(n,q)$ le plus petit polynôme unitaire de degré $n$ sur $\F_q$.

\begin{theorem}
	Pour tout w$q$ puissance d'un nombre premier et $n \geq 1$ pm a $I(n,q) > 0$.\\

	Plus précisément:
	$$ I(n,q) = \frac{1}{n} \sum_{d|n} \mu(\frac{n}{d}) q^d$$

	où $\mu$ est la fonction de Möbius:

	\[ \mu(k) = \begin{array}{ll}
			0      & \text{si il existe } l \text{ premier tel que } l^2 \text{ divise } k                            \\
			(-1)^r & \text{si } k = p_1 \cdot p_2 \cdot \ldots \cdot p_r \text{ avec } p_i \text{ premiers distincts}
		\end{array} \]

\end{theorem}

\begin{proof}
	%TODO
\end{proof}


\subsection{Critères de réductibilité sur $\Q$ et $\F_p = \Z/p\Z$}
Pour étudier la réductibilité d'un polynôme sur $\Q$ on peut toujours se ramener à un
polynôme sur $\Z$ primitif.



\begin{prop}
	Soit $P = a_nX^n + \ldots + a_0 \in \Z[X]$.\\
	Soir $p$ un nombre premier tel que $p \nmid a_n$.\\
	Si $\bar{P}$,la réduction modulo $p$, est irréductible sur $\F_p$, alors $P$ est irréductible sur $\Q$.
	De plus, si $P$ est primitif, alors $P$ est irréductible sur $\Z$.
\end{prop}


\begin{remarque}
	$a \nmid b$ est essentiel : $2X^2 - X + 1$ est irréductible sur $\F_2$ mais réductible sur $\Q$, 1 est une racine.
\end{remarque}

\begin{remarque}
	Ce critère est une condition suffisante mais pas nécessaire.\\
\end{remarque}


\begin{proof}
	On suppose $P$ primitif, réductible sur $\Q$ et donc aussi sur $\Z$.\\
	Alors il existe $R, S \in \Q[X]$ , tels que $P = RS$
	$$ \exists a, b \in \N* \text{ tels que } aR, bS \in \Z[X]$$
	$$ abP = aRbS$$
	$$ c(abP) = ab = c(aR)c(vS) $$
	$$P =   \frac{c(aR)}{c(aR)} \frac{c(bS)}{c(bS)}$$
	Modulo $p$ $\bar{P} = \bar{Q}\bar{R}$.
	De plus, on a:
	\begin{itemize}
		\item $\deg Q = \deg \bar{Q}$
		\item $\deg R = \deg \bar{R}$

	\end{itemize}
	car leurs coefficients dominants ne divisent pas $p$.\\
	Donc $\bar{P} = \bar{Q}\bar{R} \implies \bar{P}$ n'est pas irréductible dans $\Zn{p}[X]$.\\
	On a donc montré la contraposée de la proposition.
\end{proof}


\begin{prop}
	Soit $P \in K[X]$ et $\deg P = n$ n'a pas de racines dans toute extension
	de $K$ de degré au plus $\frac{n}{2}$, alors $P$ est irréductible sur $K$.
\end{prop}

\begin{remarque}
	Si $n = 2$ ou $n = 3$ ce résultat dit que si $P$ n'a pas de racines dans $K$, alors $P$ est irréductible.
\end{remarque}


\begin{proof}
	Soit $P$ réductible sur $K$. Alors il existe $Q$ irréductible de degré $\leq \frac{n}{2}$ qui divise $P$.\\
	Soit un corps de rupture pour $Q$ sur $K$.\\
	$$ [L : K] = \deg Q \leq \frac{n}{2}$$
	$L$contient une racine de $Q$ et donc une racine de $P$.
\end{proof}


%TODO





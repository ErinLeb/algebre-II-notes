\subsection{Critères de réductibilité sur $\Q$ et $\F_p = \Z/p\Z$}
Pour étudier la réductibilité d'un polynôme sur $\Q$ on peut toujours se ramener à un
polynôme sur $\Z$ primitif.



\begin{prop}
	Soit $P = a_nX^n + \ldots + a_0 \in \Z[X]$.\\
	Soir $p$ un nombre premier tel que $p \nmid a_n$.\\
	Si $\bar{P}$,la réduction modulo $p$, est irréductible sur $\F_p$, alors $P$ est irréductible sur $\Q$.
	De plus, si $P$ est primitif, alors $P$ est irréductible sur $\Z$.
\end{prop}


\begin{remarque}
	$a \nmid b$ est essentiel : $2X^2 - X + 1$ est irréductible sur $\F_2$ mais réductible sur $\Q$, 1 est une racine.
\end{remarque}

\begin{remarque}
	Ce critère est une condition suffisante mais pas nécessaire.\\
\end{remarque}


\begin{proof}
	On suppose $P$ primitif, réductible sur $\Q$ et donc aussi sur $\Z$.\\
	Alors il existe $R, S \in \Q[X]$ , tels que $P = RS$
	$$ \exists a, b \in \N* \text{ tels que } aR, bS \in \Z[X]$$
	$$ abP = aRbS$$
	$$ c(abP) = ab = c(aR)c(vS) $$
	$$P =   \frac{c(aR)}{c(aR)} \frac{c(bS)}{c(bS)}$$
	Modulo $p$ $\bar{P} = \bar{Q}\bar{R}$.
	De plus, on a:
	\begin{itemize}
		\item $\deg Q = \deg \bar{Q}$
		\item $\deg R = \deg \bar{R}$

	\end{itemize}
	car leurs coefficients dominants ne divisent pas $p$.\\
	Donc $\bar{P} = \bar{Q}\bar{R} \implies \bar{P}$ n'est pas irréductible dans $\Zn{p}[X]$.\\
	On a donc montré la contraposée de la proposition.
\end{proof}


\begin{prop}
	Soit $P \in K[X]$ et $\deg P = n$ n'a pas de racines dans toute extension
	de $K$ de degré au plus $\frac{n}{2}$, alors $P$ est irréductible sur $K$.
\end{prop}

\begin{remarque}
	Si $n = 2$ ou $n = 3$ ce résultat dit que si $P$ n'a pas de racines dans $K$, alors $P$ est irréductible.
\end{remarque}


\begin{proof}
	Soit $P$ réductible sur $K$. Alors il existe $Q$ irréductible de degré $\leq \frac{n}{2}$ qui divise $P$.\\
	Soit un corps de rupture pour $Q$ sur $K$.\\
	$$ [L : K] = \deg Q \leq \frac{n}{2}$$
	$L$contient une racine de $Q$ et donc une racine de $P$.
\end{proof}


%TODO




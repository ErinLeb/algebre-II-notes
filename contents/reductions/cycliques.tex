\subsection{Endomorphismes cycliques}

Sot $u \in \LE$ et $x \in E$ non nul.
Le sous espace vectoriel cyclique $E_{u,x}$ est $e$ plus petit sous espace vectoriel de $E$ contenant $x$ et stable par $u$.

Il est engendré par les $u^k(x), \ k \in \N$ et si la dimension de cet espace est $p$, alors la famille
$ (x, u(x), \ldots, u^{p-1}(x))$ est une base de $E_{u,x}$.

\begin{proof}
	\begin{itemize}
		\item Soit $F \subset vect (x, u(x), \ldots, u^{p-1}(x))$.\\
		      $x \in F$ et $F$ est stable par $u$.\\
		      Donc $E_{u,x} \subset F$.

		\item Réciproquement, les $u^k(x) \in E_{u,x}$ car $x \in E_{u,x}$ et $E_{u,x}$ est stable par $u$.
		      Donc $F \subset E_{u,x}$.
	\end{itemize}
	On en conclut que $E_{u,x} = vect (x, u(x), \ldots, u^{p-1}(x))$.
\end{proof}

On désigne par $\mu_{u,x}$ le polynôme minimal de $u$ en $x$, c'est à dire le générateur unitaire de l'idéal
$$ \set{P \in K[X] \mid P(u)(x) = 0_E}$$

Comme $\dim E$ est fini cet idéal est \textbf{non} réduit à $\set{0}$.


Si $\mu_{u,x} = X^d + \sum_{k=0}^{d-1} a_k X^k$ alors la famille $(x, u(x), \ldots, u^{d-1}(x))$ est libre.

La famille est génératrice.\\
En effet, si on écrit $v = \sum \beta_k u^k(x) = P(u)(x)$, avec $P = \sum \beta_k X^k$.

On fait la division euclidienne de $P$ par $\mu_{u,x}$:

$$ P = Q \mu_{u,x} + R, \ \deg R \leq d -1$$
%TODO Improve this
$$ v = P(u)(x) = Q(u) \mu_{u,x}(u)(x) + R(u)(x) = R(u)(x)$$

Car $\mu_{u,x}(u)(x) = 0$ par définition.

$$ P(u)(x) \in vect (x, u(x), \ldots, u^{d-1}(x))$$

On a aussi $(u^k(x))_{k \in \N}$ base de $E_{u,x}$, donc en particulier, $ d = \dim E_{u,x} - p $.

\begin{definition}
	$u \in \LE$ est dit cyclique \ssi $\exists x \in E$ tel que $E = E_{u,x}$.
\end{definition}

\begin{example}
	Si $\dim E = 2$ $u$ est soit une homothétie, soit un endomorphisme cyclique.

	\begin{itemize}
		\item Soit pour tout $x \in E$ non nul, $\dim (x, u(x)) = 1$.

		      $\forall x \in E, \exists \lambda_x \in K, u(x) = \lambda_x x$.

		      \begin{itemize}
			      \item $\lambda_x$ ne dépend pas de $x$. En effet, si $x$ et $y$ sont non colinéaires:
			            $$ u (x+y) = \lambda_{x+y} (x+y) = \lambda_x x + \lambda_y y \implies (\lambda_{x+y} - \lambda_x) x = (\lambda_{x+y} - \lambda_y) y$$
			            Donc $\lambda_{x+y} = \lambda_x = \lambda_y$.

			      \item Si $x$ et $y$ sont colinéaires, alors $\exists z$ non colinéaire à $x$ et donc à $y$. D'après ce qui précède, $\lambda_x = \lambda_z = \lambda_y$.
		      \end{itemize}

		      Donc $\lambda_x = \lambda$ ne dépend pas de $x$, et donc $u$ est une homothétie.

		\item Sinon $\exists x \in E$ tel que $(x, u(x))$ est une base de $E$ et dans ce cas $u$ est cyclique
	\end{itemize}
\end{example}

\begin{example}
	Soit $u \in \LE$, $\dim E = n$, On suppose que $u$ admet $n$ valeurs propres distinctes. Alors $u$ est cyclique.

	Soit $x_j$ un vecteur propre associé à la valeur propre $\lambda_j$.

	La matrice de passage entre $(x_1, \dots, x_n)$ et $(x, u(x), \dots, u^{n-1}(x))$ est une matrice de Vandermonde.

	$u(x) = \lambda_1^k x_1 + \dots + \lambda_n^{k}x_n$
	$u^k(x) = \lambda_1^k x + \dots + \lambda_n^{k}x$

	\begin{equation*}
		M = \begin{pmatrix}
			1      & \lambda_1 & \lambda_1^2 & \cdots & \lambda_1^{n-1} \\
			1      & \lambda_2 & \lambda_2^2 & \cdots & \lambda_2^{n-1} \\
			\vdots & \vdots    & \vdots      &        & \vdots          \\
			1      & \lambda_n & \lambda_n^2 & \cdots & \lambda_n^{n-1}
		\end{pmatrix}
	\end{equation*}

	On a que $\det M = 0 $ \ssi les $\lambda_j$ sont disjoints et
	$$ \det M = \prod_{i < j} (\lambda_j-\lambda i)$$
\end{example}

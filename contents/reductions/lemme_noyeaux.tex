\subsection{Lemme des noyaux}

\subsubsection{Etude du $\ker P(u)$}


\begin{prop}
	Soient $P,$ et $Q \in K[X]$ tels que $\pgcd{P}{Q} = D$ et $u \in \LE$.
	Alors
	$$\ker P(u) \cap \ker Q(u) = \ker D(u).$$
\end{prop}

\begin{proof}
	\begin{itemize}
		\item $D \mid P \implies \ker D(u) \subset \ker P(u)$\\
		      En effet, $P = RD$, $P(u)(x) = R(u)(D(u)(x)) = 0$.\\
		      Donc $D(u)(x) = 0 \implies P(u)(x) = 0$
		      $$ \ker D(u) \subset \ker P(u) \cap \ker Q(u).$$

		\item Montrons l'autre inclusion.\\
		      Le théorème de Bézout affirme l'existence de $U$ et $V \in K[X]$ tels que $U(X)P(X) + V(X)Q(X) = D(X)$.\\
		      Donc $\forall x \in E$,
		      $$ D(u)(x) = U(u)P(u)(x) + V(u)Q(u)(x) = 0$$
		      Donc $x \in \ker P(u) \cap \ker Q(u) \implies x \in \ker D(u)$.
	\end{itemize}
\end{proof}

\begin{coro}
	Soit $u \in \LE$ de polynôme minimal de $\mu_u$.
	Soit $P \in K[X]$.
	$$\ker P(u) = \ker D(u)$$
	Où $D = \pgcd{P}{\mu_u}$.
\end{coro}

\begin{remarque}
	Ce résultat nous permet de nous restreindre a des $P$ tels que $P \mid \mu_u$.
\end{remarque}

%TODO: add example showing that \mu_u in not necessarily irreducible

\begin{proof}
	Par définition $\ker \mu_u(u) = E$, car $\mu_u(u) = 0_{\LE}$, donc
	\begin{eqnarray*}
		\ker P(u)  &=& \ker P(u) \cap E \\
		&=& \ker P(u) \cap \ker \mu_u(u) \\
		&=& \ker D(u)
	\end{eqnarray*}
\end{proof}

\begin{coro}
	Soit $u \in \LE$ de polynôme minimal $\mu_u$ et $P \in K[X]$ unitaire, avec $P \mid \mu_u$.\\
    Soit $v = \restr u {\ker P(u)}$.\\
	Alors $\mu_v = P$.
\end{coro}


\begin{proof}
	Pour tout $P$, $\ker P(u) $ est stable par $u$.\\
	En effet, si $x \in \ker P(u)$, alors $P(u)(x) = 0$, donc
	\begin{eqnarray*}
		P(u)(u(x)) &=& (P (u) \circ u)(x) \\
		&=& (u \circ P(u))(x) \\
		&=& u(P(u)(x)) \\
		&=& 0_{\LE}
	\end{eqnarray*}
	Donc $u(x) \in \ker P(u)$.\\ %TODO: Maybe add remark about (XP(X))(u)...

	Montrons que $F = \ker P(u), \ \forall x \in F, \ P(v(x)) = P(u)(x) = 0$.\\

	$P(0)=0_{\LE}$, $\mu_v \mid P$.\\
	Prenons $Q \in K[X]$ tel que $\mu_v = QP$.\\
	$\mu_v(u) = 0_{\LE} \implies \ker Q(u) \subset \ker P(u)$.\\
	%TODO: Finish
\end{proof}



\subsubsection{Lemme des noyaux}

Soit $u \in \LE$.\\
Il permet de décomposer un espace vectoriel en some directe de espaces vectoriels stables par $u$ et adaptés à la réduction de $u$.

\begin{lemma}[des noyaux]
	Soit $(P_k)$ une famille de polynômes 2 à deux premiers entre eux.
	Soit $u \in \LE$.
	$$ \ker \left(\left(\prod_{k=1}^N P_k \right)(u)\right) = \bigoplus_{k=1}^N \ker P_k(u)$$

	De plus, la projection de $\ker \left(\left(\prod_{k=1}^N P_k \right)(u)\right)$ sur l'un des  $\ker P_j(u)$ parallèlement 'a la somme des autres est un polynôme en $u$.
\end{lemma}

\begin{proof}
	Par récurrence sur $N$.

	\begin{itemize}
		\item Montrons que si $P_1$ et $P_2$ sont premiers entre eux, alors $\ker (P_1P_2) (u) = \ker P_1(u) \oplus \ker P_2(u)$.\\
		      %TODO
	\end{itemize}
\end{proof}


\subsubsection{Conséquences: Décomposition en sous espaces vectoriels stables}


Soir $u \in \LE$ avec un polynôme minimal $\nu_u$. On décompose $\nu_u$ en produit
d'irréductibles:
$$ \nu_u = \prod\limits_{k=1}^N p_k^{\alpha_k} $$
où $\alpha_k \geq k$, les $p_k$ sont irréductibles unitaires et 2 à 2 distincts.


\begin{coro}
	$$E = \bigoplus_{k=1}^N \ker\left( {P_k^{\alpha_k}(u)} \right)$$
\end{coro}

\begin{proof}
	La preuve consiste a utiliser le lemme des noyaux sachant que
	$$\ker \mu_u(u) = E $$. Les sous espaces vectoriels  $\ker\left( {P_k^{\alpha_k}(u)} \right)$ sont stables par $u$.
	Il suffit pour réduire $u$ de se restreindre à chacun des ces sous espaces vectoriels.
\end{proof}


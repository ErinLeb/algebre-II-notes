\subsection{Rappels d'algèbre linéaire}

\subsubsection{Déterminants} %Maybe don't use subsubsections


Soit $n = \dim E$, pour tout $u \in \LE$, on peut associer $\det u \in K$.

Il y a plusieurs façons de le définir.


\begin{itemize}
	\item De manière canonique : pour toute forme $n$ linéaire alternée $f$ sur $E$ on a:
	      $$ f \circ u = \det (u) * f$$
	      En effet, l'espace de ces formes $n$ linéaires alternées est de dimension 1. De cette manière, $f$ et $f \circ u$ sont proportionnelles.

	      On qualifie cette définition de canonique car elle ne nécessite pas le choix d'une base de $E$.

	\item Soit de manière matricielle : $\det u$ est un polynôme explicite en les coefficients de la matrice de $U$ prise dans une base de $E$.
	      Ce déterminant ne dépend pas du choix de la base.
\end{itemize}


D'après 1, $det (Id_E) = 1$.

Si $u$ et $v \in \LE$,
\begin{eqnarray*}
	f \circ u \circ v &=& \det (u \circ v) f \\
	&=& \det (v) f  \circ u \\
	&=& \det (v) \det (u) f
\end{eqnarray*}

Alors,
$$\det (u \circ v) = \det (u) \det (v) = \det (v \circ u) $$

En particulier, si $v$ est inversible, alors
$$ \det (v^{-1}) = (\det v )^{-1} $$

Le déterminant induit un morphisme de groupe :


\begin{eqnarray*}
	\det (GL(E), \circ) &\to& (K^{\times}, *)\\
	u &\mapsto& \det (u)
\end{eqnarray*}

$u$ est inversible \ssi $det(u) \neq 0$

\subsubsection{Polynôme caractéristique d'un endomorphisme}

On apelle polynôme caractéristique de $u$ le polynôme  $\chi_u$ défini par
$$\chi_u(X) = \det (X Id_e - u)$$

On fixe une base $B$ de $E$, on définit $M = Mat_B u$.
Dans ce cas $\chi_u(X) = \det (CI_n -M)$
(c'est un polynôme en $X$ de degré $n$ unitaire).

La définition de $\chi_u(X) = \det (XI_n-M)$ ne dépend pas du choix de la base.
Si $P$ inversible, alors:


\begin{eqnarray*}
	\det (XI_n - PMP^{-1}) &=& \det (P(XI_n -M) P^{-1})\\
	&=& \det (P) \det (X-I_nM)\det P^{-1}\\
	&=& \det (XI_n -M)
\end{eqnarray*}



De cette manière, si $v$ inversible:
$$ \chi_{v\circ u \circ v^{-1}} = \chi_{u}$$


\subsubsection{Notion de valeur propre}

$\lambda \in K$ tel que $\exists x \in E$ non nul vérifiant $u(x) = \lambda x$.

\subsubsection{Notion de vecteur propre associé à une valeur propre}

$$E_n = \set{ x \in E: u (x) = \lambda x } = \ker (u - \lambda Id_E) $$

\begin{prop}
	Les vecteurs propres associés à des valeurs propres 2 à 2 disjoints forment une famille libre.
\end{prop}

\begin{coro}
	Soit $E$ de dimension finie $n$ et $u \in \LE$, alors $u$ a au plus $n$ valeurs propres distinctes.
\end{coro}


\begin{prop}
	Les sous-espaces vectoriels correspondant à deux valeurs propres distinctes sont en somme directe.

	Autrement dit,
	$$E_{\lambda_1} \cap E_{\lambda_2} = \set{0} \text { si } \lambda_1 \neq \lambda_2$$
\end{prop}

\begin{prop}
	Soit $u \in \LE$ et $P \in K[X]$: Pour toute valeur propre $\lambda$ de u:
	$$ P(u) = 0_{\LE} \implies P(\lambda) = 0_K$$
\end{prop}

\begin{proof}
	Si $x \in E_\lambda , P(u)(x) = P(\lambda) x$ %TODO: add yellow explanation
\end{proof}


\begin{prop}
	Soit $u \in \LE$ de polynôme minimal $\nu_u$,
	$$ \lambda \text{ valeur propre de u }\iff \mu_u(\lambda) = 0$$
\end{prop}


\begin{proof}
	\begin{itemize}
		\item  $\Rightarrow$:
		      $\lambda$ valeur propre de $u$ (on dit aussi $\lambda \in \spec(u)$)
		      $$\exists x \in E \text { non nul tel que } u(x) = \lambda x \ \nu_u(u)(x) = \nu_u(\lambda)x =0$$
		\item $\Leftarrow$:
		      $\nu_u(\lambda) = 0 \implies \exists Q \in K[X]$ tel que $\nu_u(X) = (X - \lambda)Q(X)$\\
		      Si $\lambda$ n'était pas une valeur propre, alors $u-\lambda Id_E$ serait inversible
		      $$ (u-  \lambda Id) \circ Q = 0 \iff Q(u) = 0_{\LE}$$
		      Ce qui contredit la minimalité de $\nu$.
	\end{itemize}
\end{proof}



\begin{prop}
	Soit $u \in \LE$, $E$ de dimension $n$.
	$F$ un sous-espace vectoriel de $E$ stable par $u$.
	On peut définir $v \in \LE$ par $v = \restr u F$. On a alors:
	$$\chi_v \text{ divise } \chi_u$$
\end{prop}

\begin{proof}
	%TODO
\end{proof}

\begin{prop}
	Soit $u \in \LE$ et $E = F \oplus G$ avec $F$ et $G$ stables par $u$ avec $n = \dim E$.

	$$\chi_u (X)= \chi_v (X) \chi_w(X)$$
	avec $v = u_{|F}$ et $w = u_{|G}$
\end{prop}

\begin{proof}
	La même que précédemment, on posant $(f_{p+1}, f_n)$ une base de $E$.
	Dans ce cas-là, $B = 0$, $D = mat_{(f_{p+1}, \dots, f_n)} w$
	$det(XI_{n_p} - D) = \chi_w(X)$
\end{proof}


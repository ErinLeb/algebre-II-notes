\subsection{Critères d'irréductibilité dans $K[X]$}

\begin{prop}
	Un polynôme de degré 1 dans $K[X]$ est irréductible dans $K[X]$.
\end{prop}

\begin{proof}
	Si $P = QR$, on a $ 1 = \deg(P) = \deg(Q) + \deg(R)$, donc $Q$ ou $R$ est de dégré 0, donc c'est un inversible.
\end{proof}

\begin{theorem}[d'Alambert-Gauss]
	Les irréductibles de $\CX$ sont les polynômes de degré 1.
\end{theorem}

\begin{proof}
	Tout polynôme $P\in \CX$ de degré $\geq 1$ admet $\alpha \in \C$ racine et donc $X-\alpha$ divise $P$.
\end{proof}

\begin{prop}
	Soit $P \in \mathbb{R}[X]$ irréductible alors:
	\begin{itemize}
		\item Soit $\deg P = 1$.
		\item Soit $\deg P = 2$ et $P$ n'admet pas de racines dans $\mathbb{R}$.
	\end{itemize}
\end{prop}

\begin{proof}
	$P(\alpha) = 0 \implies P(\bar{\alpha}) = 0$
	sur $\C \setminus \R, \, (X- \alpha) (X- \bar{\alpha}) = X ^2 - (\alpha + \bar{\alpha})X +
		\alpha\bar{\alpha} \in \RX$
	et divise $P$.
\end{proof}

\begin{prop}
	Soit $P \in K[X]$ de degré 2 ou 3,
	$P$ est irréductible dans $K[X]$ $\iff$ $P$ n'admet pas de racines dans $K$.
\end{prop}

\begin{example}[contre-exemple]
	$K = \R, P = (X^2 + 1 )^2$ n'a pas de racines dans $\R$, mais il n'est pas irréductible $(\deg P = 4 > 3)$.
\end{example}

\begin{example}
	$k = \Q, \,p = (x^2 + 2 )$ n'a pas de racines, donc irréductible dans $\QX$, mais il a des racines sur $\RX$, donc pas irrédutible sur $\RX$.
\end{example}

\begin{theorem}[Critère d'Eisenstein]
	Soit $A$ un anneau factoriel, $P$ un polynôme  de $\AX$ de degré $\geq 1$, $p$ un irréductible de $A$.
	On écrit $P = \sum\limits_{i=0}^n a_iX^i$ avec $a_i \neq 0$. Si on a les trois propriétés suivantes:
	\begin{itemize}
		\item $p$ ne divise pas $a_n$
		\item $p$ divise pas $a_k \ \forall \ k < n$
		\item $p^2$ ne divise pas $a_0$
	\end{itemize}
	Alors $P$ est irréductible dans $K[X]$(avec $K$ corps des fractions).
\end{theorem}

\begin{example}
	$P(X) = X^5 + 2X^4 + 2024X + 6\in \QX$. On a que $P$ est 2-eisenstein et donc irréductible.
\end{example}

\begin{coro}
	Si de plus $P$ est primitif de $\AX$, alors $P$ est irréductible dans $A[X]$.
\end{coro}


\begin{proof}
    Quitte à diviser par $c(P)$, on peut supposer $P$ primitif et de degré $\geq 2$.
	Si $P$ n'est pas irréductible, il s'écrit $P=RQ$, avec $R, Q \in A[X]$ de degré $>0$.
	On écrit $$Q = b_sX^s + \cdots + b_0$$ et $$R = c_rX^r + \cdots + c_0$$
	Soit $B = A/<p>$ intègre et on a $A[X]/pA[X] \simeq B[X]$.
	Dans $B[X]$, on a $\bar{P} = \bar{R}\bar{Q}$,
	Or d'après les hypothèses on a $\bar{P} = \bar{a}_0X^n$ et $\bar{a_n} \neq 0 $ dans $B$.
	Donc $\bar{b_s} \neq 0$ et $\bar{c_r} \neq 0$ et $\bar{b_s}\bar{c_r} = \bar{a_n}$ dans $B$,
	et $\bar{Q}$ et $\bar{R}$ sont de degré $>0$ et $\bar{Q}\bar{R} = \bar{a_n}X^n$ dans $B[X]$.

	On voit la relation $\bar{a_n}X^n= \bar{Q}\bar{R}$ dans $B[X]$ qui est principal et donc factoriel.


    $\bar{Q}$ et $\bar{R}$ sont des polynomes irréductibles dans $(Frac B)[X]$. Donc en particulier
	$p$ divise $b_0$ et $c_0$, donc $p^2$ divise $b_0c_0 = a_0$, ce qui est absurde.
\end{proof}


\begin{example}
	$A = \Z$, $p$ premier,\\
	$Q \in \Z[X] \implies \bar{Q} \in \Zn{p}[X]$.\\
	$\bar{Q}$ est défini par la classe $mod\  p$ de ses coefficients.
	$$X \mid \bar{Q} \text{ dans } \Zn{p}[X] \iff p \mid Q(0) \text{ dans } \Z$$
	\begin{eqnarray*}
		\Z[X] &\to& \Zn{p}[X] \\
		\sum a_iX^i &\mapsto& \sum \bar{a_i}X^i
	\end{eqnarray*}
\end{example}


\begin{example}
	$\Phi_p(x) = x^{p-1} + \cdots + x + 1  = \frac{x^p - 1}{x-1} \in \Z[X]$.
	On applique le critère d'Eisenstein à $\Phi_p(X+1)$.
	$$ \Phi_p(X+1) = \frac{(X+1)^p - 1}{X} = \sum_{k=1}^p \binom{p}{k} X^{k-1} = \sum_{k=0}^{p-1} \binom{p}{k+1} X^k$$
	\begin{itemize}
		\item Le coéfficient dominant de $X^{p-1}$ est $\binom{p}{p} = 1$ et $p$ ne divise pas 1.
		\item Pour tout $k < p-1$, le coefficient de $X^k$ est $\binom{p}{k+1} = \frac{p!}{(k+1)!(p-k-1)!} = p \underbrace{\frac{(p-1)!}{(k+1)!(p-k-1)!}}_{\in A}$.
		      $p$ divise $p! = k!(p-k)!\binom{p}{k}$ et $p$ premier à $k!(p-k)!$, donc $p$ divise $\binom{p}{k}$.
		\item $\binom{p}{1} = p$ et $p^2$ ne divise pas $p$. donc
	\end{itemize}
	Donc $\Phi_p(X+1)$ est irréductible dans $\Q[X]$ et donc $\Phi_p(X)$ est irréductible dans $\Q[X]$.
\end{example}


\begin{prop}
	Soi $P \in \Z[X]$ primitif de coefficient dominant non multiple de $p$, où $p$ est un premier.
	Si $\bar{P}$ est irréductible dans  $\Zn{p}[X]$, alors $P$ est irréductible dans $\Z[X]$.
\end{prop}

\begin{proof}
    $P$ est primitif, donc si P est est non irrédutible dans $\Z[X]$ alors il existe $Q, R \in \Z[X]$ non constants, tels que $P = QR$.
	et donc $\bar{P} = \bar{Q}\bar{R}$ dans $\Zn{p}[X]$.
	De plus, on a:
	\begin{itemize}
		\item $\deg Q = \deg \bar{Q}$
		\item $\deg R = \deg \bar{R}$

	\end{itemize}
	car leur coefficients dominant ne divisent pas $p$.
	Donc $\bar{P} = \bar{Q}\bar{R} \implies \bar{P}$ n'est pas irréductible dans $\Zn{p}[X]$ ce qui contredit l'hypothèse.
\end{proof}

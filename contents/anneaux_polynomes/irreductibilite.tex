\subsection{Critères d'irréductibilité dan $K[X]$}


\begin{prop}
	Un polynôme de degré 1 dans $K[X]$ est irréductible dans $K[X]$.
\end{prop}

\begin{proof}
	Si $P = QR$ on a $ 1 = \deg(P) = \deg(Q) + \deg(R)$, donc $Q$ ou $R$ est de dégré 0, donc c'est un inversible.
\end{proof}:


\begin{theorem}[d'Alambert-Gauss]
	Les irréductibles de $\mathbb{C}[X]$ sont les polynômes de degré 1.
\end{theorem}


\begin{proof}
	Tout polynôme $P\in \mathbb{C}[X]$ de degré $\geq 1$ admet $\alpha \in \mathbb{C}$ racine et donc $X-\alpha$ divise $P$.
\end{proof}



\begin{prop}
	Soit $P \in \mathbb[R][X]$ irréductible alors:
	\begin{itemize}
		\item Soit $\deg P = 1$.
		\item Soit $\deg P = 2$ et $P$ n'admet ps de racines dans $\mathbb{R}$.
	\end{itemize}
\end{prop}


\begin{proof}
	$P(\alpha) = 0 \implies P(\alpha) = 0$ %TODO overline
	Sur $\mathbb{C} \setminus \mathbb{R}, \, (X- \alpha) (X- \alpha) = X ^2(\alpha + \alpha)X + \alpha\alpha \in \mathbb{R}[X]$ %TODO overline
	et divise $P$.
\end{proof}



\begin{prop}
	Soit $P \in K[X]$ de degré 2 ou 3,
	$P$ est irréductible dans $K[X]$ $\iff$ $P$ n'admet pas de racines dans $K$.
\end{prop}


\begin{exemple}[contre-exemple]
	$K = \mathbb{R}, P = (X^2 + 1 )^2$ n'a pas de racines dans $\mathbb{R}$, mais iln'est pas irréductible $(\deg P = 4 > 3)$.
\end{exemple}

\begin{exemple}
    $k = \mathbb{q}, p = (x^2 + 2 )$ n'a pas de racines, donc irréductible dans $\mathbb{q}[x]$, mais il a des racines sur $\mathbb{r}[x]$, donc pas irrédutible sur $\mathbb{r}[x]$.
\end{exemple}


\begin{theorem}[Critère d'Eisenstein]
    Soit $A$ un anneau factoriel, $P$ un polynôme  de $A[X]$ de degré $\geq 1$, $p$ un irréductible de $A$.
    On écrit $P = \sum_{i=0}^n a_iX^i$ avec $a_i \neq 0$ si on a les trois propriétés suivantes: 
    \begin{itemize}
        \item $p$ ne divise pas $a_n$
        \item $p$ divise pas $a_k \forall k < n$
        \item $p^2$ ne divise pas $a_0$
    \end{itemize}
    Alors $P$ est irréductible dans $K[X]$(avec $K$ corps des fractions).
\end{theorem}

\begin{exemple}
    $P(X) = X^5 + 2X^4 + 2024X + 6\in \mathbb{Q}[X]$. On a que $P$ est 2-eisenstein et donc irréductible.
\end{exemple}



\begin{coro}
    Si de plus $P$ est primitif de $A[X]$, alors $P$ est irréductible dans $A[X]$.
\end{coro}

\subsection{Construction formelle}

\begin{definition}[Anneau de polynômes]

	Soit $A$ un anneau commutatif. On note $A^{\mathbb{N}}$ l'ensemble des suites presques nulles d'éléments de $A$.
	On muni $A^{\mathbb{N}}$ d'une structure d'anneau en posant:
	\begin{itemize}
		\item $(a_n)_{n \in \mathbb{N}} + (b_n)_{n \in \mathbb{N}} = (a_n + b_n)_{n \in \mathbb{N}}$
		\item $(a_n)_{n \in \mathbb{N}} \cdot (b_n)_{n \in \mathbb{N}} = (c_n)_{n \in \mathbb{N}}$ où $c_n = \sum_{k=0}^{n} a_k b_{n-k}$
	\end{itemize}

	On note $A[X]$ l'anneau commutatif (proposition a montrer si besoin)  $A^{\mathbb{N}}$.

\end{definition}

\begin{definition}[Anneau intègre]
	Un anneau $(A, +, \cdot)$ est intègre s'il est commutatif, non trivial et pour tout $x, y \in A$,
	\begin{equation*}
		xy = 0 \implies x = 0 \ \text{ ou } \ y = 0
	\end{equation*}
\end{definition}

\begin{prop}
	Soit $A$ un anneau commutatif.
	Soient $P, Q \in A[X]$. Alors $\deg(PQ) \leq \deg(P) + \deg(Q)$.
	De plus, si $A$ est intègre, $\deg(PQ) = \deg(P) + \deg(Q)$.
\end{prop}

\begin{proof}
	Soient $P=a_n x^n+\ldots+a_0$ et $Q=b_m x^m+$ $\cdots+b_0$ avec $a_n \neq 0$ et $b_m \neq 0$.
	Ainsi, $\deg(P)=n$ et $\deg(Q)=m$. Le terme de plus haut degré dans $P Q$ vient de $a_n X^n \cdot b_m X^m=a_n \cdot b_m X^{n+m}$.
	Par conséquent, $\deg(P Q) \leq n+m=\deg(P)+\deg(Q)$.

	Si A est intègre, alors $a_n b_m \neq 0$ si $a_n \neq 0$ et $b_m \neq 0$. Ainsi:
	$$
		\deg(P Q)=n+m=\deg(P)+\deg(Q).
	$$
\end{proof}

\begin{coro}
	$A[X]$ intègre $\iff A$ intègre.
\end{coro}

\begin{proof}
	\begin{itemize}
		\item $\Rightarrow$ C'est immédiat, car un sous-anneau d'un anneau intègre est intègre.\\
		\item $\Leftarrow$ Soient $P, Q \in A[X]$ non nuls. on a donc $\deg(P) \geqslant 0$ et $\deg(Q) \geqslant 0$.\\
		      Mais alors, on a :
		      $$ \deg(P Q)=\deg(P)+\deg(Q) \geqslant 0, \text{par la proposition précédente} $$
		      Ce qui entraîne que $P Q$ est non nul, d'où l'intégrité de $A[X]$.
	\end{itemize}
\end{proof}

\begin{example}
	$\F_p=\Zn{p}$, avec $p$ premier, est intègre, donc $\F_p[X] = \Zn{p}[X]$ est intègre
\end{example}

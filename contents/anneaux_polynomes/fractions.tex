\subsection{Fractions rationelles}

\begin{definition}
	On écrit tous les éléments de $K(X)$ sous la forme $\frac{P}{Q}$ avec $P,Q \in K[X]$ et $Q \neq 0$.
\end{definition}

\begin{theorem}[Décomposition en éléments simples]
	Soit $K$ un corps. toute fraction rationelle $F = \frac{P}{Q} \in K(X)$ admet une décomposition comme somme
	d'éléments simples, c'est-à-dire comme la somme d'un polynôme $T \in K[X]$ (appelé partie entière de F) et de fractions
	$\frac{J}{H^k}$ où $J,H \in K[X]$, $H$ irréductible, $k \geq 1$ et $\deg J < \deg H$. \\

	De plus si $Q = H_1^{k_1} \cdots H_q^{k_q}$ et $P$ et $Q$ premiers entre eux, alors

	$$ F = \frac{P}{Q} = T + F_1 + \cdots + F_q$$
	$$\text{Où } F_i = \frac{J_{i,1}}{H_i} + \frac{J_{i,2}}{H_i^2} + \cdots + \frac{J_{i,n_i}}{{H_i}^{n_i}}$$
\end{theorem}

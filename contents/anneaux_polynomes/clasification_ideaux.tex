\subsection{Clasification des ideaux premiers de $A[X]$, avec $A$ anneau principal}

\begin{example}[Anneau factoriel non principal]
	$\Z[X]$ est un anneau factoriel, mais non principal.
\end{example}

\begin{theorem}
	Soit $A$ un anneau principal. Alors les idéaux premiers non nuls de $A[X]$ sont les idéaux de la liste suivante:
	\begin{itemize}
		\item $\goatp_{\pi} = <\pi>$ où $\pi$ est un élément irréductible de $A$. %TODO: Add { saying that they are the irreducible elements of A[X]}
		\item $\goatp_f = <f>$, où $f \in A[X]$ est un polynôme irréductible de degré $\geq 1$.
		\item $\goatm_{\pi,f} = <\pi,f>$, où $\pi$ est un élément irréductible de $A$ et $f \in A[X]$ unitaire, irréductible modulo $\pi$.
	\end{itemize}
	De plus
	\begin{itemize}
		\item $\goatp_{\pi}$ n'est pas maximal.
		\item Si $A$ possede une infinité d'éléments irréductibles deux à deux disjoints non associés, alors $p_f$ est premier mais pas maximal.
		\item $\goatm_{\pi,f}$ est maximal.
	\end{itemize}
\end{theorem}


\begin{rappel}[Irréductibilité]
	\begin{itemize}
		\item
		      Si $\pi$ est un élément irréductible de $A$, alors $A/<\pi>$ est un corps parce que dans un anneau
		      principal l'idéal engendré par un élément irréductible est maximal.\\
		\item Soit  $f \in A[X]$et $\bar{f}$ la classe de $f$ dans $A/<\pi>[X] = A[X]/<\pi>$.\\
		      alors $f$ irréductuble modulo $\pi$ signifie que $\bar{f}$ est irréductible dans $A/<\pi>[X]$.
	\end{itemize}
\end{rappel}

\begin{rappel}
	Si $f$ est un polynôme primitif de $A[X]$ de coéfficient dominant inversible modulo $\pi$, et $\bar{f}$ est irréductible dans $A/<\pi>[X]$, alors $f$ est irréductible dans $A[X]$.
\end{rappel}

\begin{proof}
	\begin{itemize}
		\item Si $\pi$ est un irréductible de $A$. Soit $g \in A[X]$ et on note $\bar{g} \in (A/<\pi>)[X]$ la classe de $g$ modulo $\pi$.
		      On rappelle
		      $$ A[X]/<\pi> \cong (A[X]/<\pi>A[X])$$
		      Comme $A$ est principal, $A/<\pi>$ est un corps et donc $(A/<\pi>)[X]$ est un anneau intègre (qui n'est pas un corps car les éléments
		      inversibles sont des polynômes constants). Donc $<\pi>$ est un idéal de $A[X]$ premier mais pas maximal.
		\item Soit $f \in A[X]$ de dégré $\geq 1$ et irréductible.$A$ principal $\implies A$ factoriel $\implies A[X]$ factoriel.
		      Tout élément irréductible de $A[X]$ engendre un idéal premier. \\
		      Si de plus $A$ possède une infinité d'éléments irréductibles deux à deux disjoints non associés, alors il exites $\pi \in A$ irréductinle de $A$
		      tel que $\pi$ ne divise pas le coefficient dominant de $f$.\\
		      Alors $<f>  \subset <\pi,f>$, et c'est une inclusion stricte.\\
		      En effet, si on avait $\pi \in <f>$ alors
		      $$\exists g \in A[X]\quad \pi = fg$$ ce qui n'est pas possible quand on regarde les dégrées \\
		      Montrons que $<\pi, f> \neq A[X]$.
		      Supposons que $<\pi, f> = A[X]$, autrent dit, il existe $g,h \in A[X]$ tels que $\pi g + fh = 1$.\\
		      $ \bar{f}\bar{g} = 1$ dans $(A/<\pi>)[X]$.\\
		      Comme $\bar{f}$ est inversible dans $(A/<\pi>)[X]$ et comme $A/<\pi>$ est intègre on a que $\deg \bar{f} = 0$.\\
		      Comme $\pi$ new divise pas le coefficient dominant de $f$, on a $\deg \bar{f} = \deg f = 0$, ce qui contredit $\deg f \geq 1$.\\
		      Donc $<\pi, f> \neq A[X]$ et $\goatp_f = <f>$ est un idéal premier mais pas maximal.
		\item  Soit $\pi$ irréductble de $A$, $f$ unitaire de $A[X]$ irréductible modulo $\pi$.\\
		      $$A[X] \to A/<\pi>[X] \to (A/<\pi>[X])/<f> $$ %TODO: Add phi
		      Le morphisme $\phi$ est surjectif de noyau $<\pi,f>$.
		      $$ A[X]/\goatm_{\pi,f} \cong (A/<\pi>[X]/<\bar{f}>)$$
		      Et donc $A/<\pi>[X]$ est principal.
		      $A$ est principal donc $<\pi>$ est maximal donc $A/<\pi>$ est un corps. Comme $A/<\pi>$ est un corps donc $(A/<\pi>)[X]$ est principal.\\
		      Comme $\bar{f}$ est un irréductible de $(A/<\pi>)[X]$, on a que $A/<\pi>[X]/<\bar{f}>$ est un corps.\\
		      Donc $A[X]/\goatm_{\pi,f}$ est un corps, donc $\goatm_{\pi,f}$ est maximal.
	\end{itemize}
	\vspace{0.25cm}
	\noindent Reciproquement, on choisit $\goatp$ un idéal non nul de $A[X]$ qui est premier. \\
	$\goatp \cap A$ est un idéal de A premier.
	Comme $A$ est principal, soit $\goatp \cap A = \{0\}$, soit il eciste $\pi$ irréductible de $A$ tel que $\goatp \cap A = <\pi>$.
	\begin{itemize}
		\item Supposons que $\goatp \cap A = <\pi>$.\\
		      On prend $\bar{\goatp}$ l'image de $\goatp$ dans $A/<\pi>[X]$ qui est principal.\\
		      $\bar{\goatp}$ est in idéal premier de l'a nneau principal $A/<\pi>[X]$.\\
		      \begin{itemize}
			      \item Soit $\bar{\goatp} = \{0\} \implies \goatp = <\pi>$.
			      \item Soit $\bar{\goatp}$ est engendré par un polynôme unitaie et irréductible de $A/<\pi>[X]$.
			            $$ \bar{\goatp} = <\bar{f}> $$
			            $f$ est un polynôme unitaire de $A[X]$ tel que $\bar{f}$est irréductible. \\
			            Donc il existe $g \in \goatp$ telle que $\bar{g} = \bar{f}$.
			            $$ \exists h \in A[X] \quad f = g + \pi h k$$
			            Comme $\pi \in \goatp$a, on a $f \in \goatp$.
			            Donc $\goatm_{\pi,f} \subset \goatp \subsetneq A$
			            donc $\goatm_{\pi,f} = \goatp$.
		      \end{itemize}
		\item Supposons que $\goatp \cap A = \{0\}$.\\
		      On choisit $f$ un élément non nul de $\goatp$ de degré minimal et par hypothèse, $\deg f \geq 1$.\\
		      On écrit $f = \alpha f_0$, avec $f_0 \in A[X]$ primitif et $\alpha \in A$.\\
		      Nous avons donc que $\alpha \in \goatp$ ou $f_0 \in \goatp$. \\
		      Or $\alpha \notin \goatp$ car $\goatp \cap A = \{0\}$, donc $f_0 \in \goatp$.\\
		      Suppososns $g \in \goatp$ non nu, alors $ \deg g \geq \deg f \geq 1$.
		      On écrit $g = hf + r$, avec $h \in K[X]$, ou $K$ est le corps des fractions de $A$ et $r \in K[X]$ de degré $< \deg f_0$.
		      (Division euclidienne dans $K[X]$).
		      On a $h \neq 0$ oar minimalité du degré de $f_0$. On aurait sinon
		      $g = r$ et $\deg r < \deg f_0$, ce qui est absurde.\\
		      On choisit $d in A$ tel que $dh$ et $df_0$ soient dans $A[X]$.\\
		      $$ dr = \underbrace{d}_{\in A} \underbrace{g}_{\in \goatp} - \underbrace{dh}_{\in A[X]} \underbrace{f_0}_{\in \goatp} \in \goatp$$
		      $$ dg =f_0hd \text{ et } c(dg) = c(f_0)c(dh) = c(dh) = dc(g)$$
		      donc $h \in A[X]$ car $g = f_0h$ et donc $\goatp = <f_0>$.
	\end{itemize}
\end{proof}



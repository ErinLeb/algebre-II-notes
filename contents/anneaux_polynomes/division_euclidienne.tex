\subsection{Division euclidienne}

\begin{prop}[Division euclidienne]
	Soit $A$ un anneau commutatif. Soient $P, Q \in A[X]$ avec $Q \neq 0$ et
	$Q$ a un coefficient dominant inversible. Alors il existe un unique couple
	$(U, R) \in A[X] \times A[X]$ tel que:
	\begin{itemize}
		\item $P = UQ + R$
		\item $\deg(R) < \deg(Q)$
	\end{itemize}
\end{prop}

\begin{prop}
	Soit $K$ un corps. $K[X]$ est un anneau euclidien et donc principal.
\end{prop}

\begin{proof}
	On commence par noter que $K$ est intègre et donc $K[X]$ aussi.
	Soit $I$ un idéal non nul de $K[X]$. On note $\mathcal{P}$ l'ensemble des degrés des polynômes de $I$:
	\[\mathcal{P} = \{ \deg(P) \mid P \in I \}\]

	$\mathcal{P}$ est non vide et minoré par $0$. Donc il existe $d \in \mathcal{P}$ tel que $d$ est minimal.
	On note $Q$ un polynôme de $I$ de degré $d$. Soit $P \in I$. On réalise la division euclidienne de $P$ par $Q$:
	\[ \exists (U, R) \in K[X] \times K[X] \mid P = UQ + R \text{ et } \deg(R) < \deg(Q) \]
	On a que $R = P - UQ \in I$  car $I$ est un idéal. Comme $\deg(R) < d$ et par définition $d$ est minimal pour
	tous les éléments non nuls, on a que $R = 0$. Donc $P = UQ$ et donc $I = <Q>$.
\end{proof}


\begin{example}[Exemple pathologique]
	$\mathbb{Z}[X]$ n'est pas principal. En effet, on peut prendre $I = <2, X>$.
\end{example}

\begin{theorem}[Propriété universelle de l'anneau de polynômes]\label{thm:prop_univ_anneau_poly}
	Soient $A$ et $B$ des anneaux commutatifs. On considère $f: A \to B$ un morphisme d'anneaux.
	Soit $b \in B$. Alors il existe un unique morphisme d'anneaux
	$\tilde{f}: A[X] \to B$ tel que $\tilde{f}(X) = b$ et $\tilde{f}_{\mid A} = f$.
\end{theorem}

\begin{definition}[Evaluation]
	Soit $A$ un anneau commutatif et $a \in A$. On note $\phi_x: A[X] \to A$ le morphisme d'anneaux
	induit par l'automorphisme trivial de $A$ en utilisant la propriété universelle de l'anneau de polynômes \ref{thm:prop_univ_anneau_poly}.

	Ce morphisme correspond à l'évaluation en $x$ : $\phi_x(P) = P(x)$.
\end{definition}

\begin{definition}[Polynôme a plusieurs variables / à n indéterminés]
	Soit $A$ un anneau commutatif. On définit par récurrence sur $n \in \N^*$ l'anneau $A[X_1, \dots, X_n]$:
	\begin{itemize}
		\item $A[X_1] = A[X]$
		\item $A[X_1, \dots, X_n] = A[X_1, \dots, X_{n-1}][X_n]$
	\end{itemize}
\end{definition}


\begin{definition}[Anneau factoriel]
	Soit $A$ un anneau commutatif. On dit que $A$ est factoriel si:
	\begin{itemize}
		\item $A$ est intègre
		\item Tout élément non nul de $A$ est inversible ou est produit d'un nombre fini
		      d'éléments irréductibles ( $a = u p_1 \dots p_n$ avec $u$ inversible et $p_i$ irréductible)
		\item La décomposition est unique à l'ordre près et à l'association près.
	\end{itemize}
\end{definition}

\begin{prop}[Admis]
	Soit $A$ un anneau factoriel. Alors $A[X]$ est factoriel.
\end{prop}


\begin{prop}
	Soit $A$ un anneau intègre tel que tout élément de $A\setminus\{A^{\times}\}$ est produit d'un nombre fini d'éléments irréductibles, alors les assertions suivantes sont équivalentes:
	\begin{enumerate}
		\item $A$ est factoriel
		\item Si $p \in A$ est irréductible, alors l'idéal $<p>$ est premier
		\item Soient $a,b,c \in A\setminus\{0\}$ tels que $a \mid bc$ et $a$ et $b$ sont premiers entre eux. Alors $a \mid c$ (lemme de Gauss).
	\end{enumerate}
\end{prop}

\begin{proof}

	3) $\implies$ 2):\\
	Soit $p \in A$ irréductible. On a que $<p> \neq A$ car $p$ est irréductible, donc pas inversible.
	Si $p$ divise $ab$ et ne divise pas $a$, alors $p$ et $a$ sont premiers entre eux car $p$ est irréductible.
	Donc tout diviseur commun de $p$ et de $c$ est soit inversible soit associé à $p$. Donc $p$ divise $c$.
	Donc d'après 3), $<p>$ est premier.

	\[ ab \in <p>\quad  et \quad a \notin <b> \implies b \in <p> \implies <p> \text{ premier} \]

\end{proof}

\begin{proof}

	2) $\implies$ 1):\\
	Soit $\mathcal{P}$ un système de représentants des irréductibles.
	\begin{equation*}
		u \cdot \prod_{p \in \mathcal{P}} p^{n_q} (\text{divisible par } q^{n_q} \in <q> ) = v \cdot \prod_{p \in \mathcal{P}} p^{m_q} (\text{divisible par } q^{m_q} \in <q>)
	\end{equation*}
	\noindent

	S'il existe $q \in \mathcal{P}$ tel que $m_q > n_q$, alors $q$ divise $u \cdot \prod\limits_{p \ne q} p^{n_q}$ ($\in <q>$), ce qui n'est pas possible par 2).

\end{proof}

\begin{proof}

	1) $\implies$ 3):\\
	$A$ est factoriel. Si $a$  divise $x$, on écrit $a$, $b$, $c$ sous la forme $u \cdot \prod\limits_{p \in \mathcal{P}} p^{v_{p(x)}}$.
	On a alors $\forall p \in \mathcal{P}$, $v_p(a) \leqslant v_p(b) \leqslant v_p(c)$ car $a$ divise $bc$.
	Si $v_p(a) \geqslant b$, alors $v_p(b) = 0$.
	Pour $H_p \in \mathcal{P}$, $v_p(a) \leqslant v_p(c)$, donc $a$ divise $c$, qui vérifie 3).

\end{proof}


\begin{definition}[pgcd]
	Le plus grand commun diviseur est défini ainsi:
	$$d = pgcd (a,b),\  \text{tout diviseur de} \ a  \ \text{et de} \ b \ \text{divise} \ d$$
	et $d$ divise $a$ et $b$
\end{definition}

\begin{remarque}
	Si $K$ est un corps, il y a un seul idéal non nul, qui est $K$ et donc
	tous les $pgcd$ valent $1$.
\end{remarque}

\begin{prop}
	Si $A$ est un anneau factoriel, alors deux éléments non nuls de $A$ admettent un $pgcd$ défini à un facteur inversible près.
\end{prop}

\begin{proof}
	Soit $\mathcal{P}$ un système de représentants des irréductibles de $A$.
	On écrit
	$$ a = u \prod\limits_{p \in \mathcal{P} } p^{n_p} \ \text{où}\  u \in A^\times $$
	$$ b = v \prod \limits_{p \in \mathcal{P} } p^{m_p} \ \text{où}\  v \in A^\times $$
	$$ pgcd(a,b)= \prod \limits_{p \in \mathcal{P}} p^{\min(m_p, m_p)}\  \text{où} \  u \in A^\times $$
	à facteur $\omega \in A^\times$ près
\end{proof}

\begin{example}
	$$A  = \mathbb{Z}, \quad  pgcd(-6,2) = 2\ \text{ou}\ -2$$
\end{example}

\begin{theorem}[admis]
	$A$ principal $\implies A$ factoriel
\end{theorem}

\begin{example}[Anneau factoriel non principal]
	$\Z[X]$ est un anneau factoriel, mais non principal.
\end{example}

\begin{prop}
	Dans un anneau principal on écrit
	$$ <a,b> \ = \  <d>, \quad \text{où}\quad d = pgcd (a,b) $$
\end{prop}

\begin{definition}
	Soit $A$ un anneau factoriel, et $P \in A[X]$, le contenu (notée $c(P)$) d'un polynôme $P$ est le
	$pgcd$ de ses coefficients non nuls.
	$P$ est dit primitif si $c(P)=1$ ( ou $c(P) \in A^\times$)
\end{definition}


\begin{example}
	$$A  = \mathbb{Z}, \quad  c(3X + 2) = 1$$
	$$A  = \mathbb{Z}, \quad  c(14X^2 + 24X + 2) = 2$$
\end{example}


\begin{lemma}[Lemme de Gauss]
	Pour tout $P,Q \in A[X]$ on a
	$$ c(PQ) = c(P)c(Q)$$
	à facteur inversible près.
\end{lemma}

\begin{proof}
	Commençons par montrer que $P$ et $Q$ primitifs implique $PQ$ primitif. \\
	Sinon, il existe un irréductible $p \in A $ tel que $p$ divise tous les coefficients de $PQ$. \\
	Supposons que $P$ et $Q$ sont primitifs. On pose $P = \sum a_iX^i$ et  $Q = \sum b_jX^j$
	On a que $$D = \{i \,\mid \, p\  \text{ne divise pas}\  a_i \}$$ n'est pas vide, car si $D$ est vide alors $ \forall\  i,\, p \,|\, a_i \implies p | c(P)$.
	On note $i_0$ (resp. $j_0$) l'indice minimal tel que $a_{i_0}$ (resp. $b_{j_0}$) ne soit pas divisible par $p$ et :
	\begin{eqnarray*}
		\forall i, \, 0 \leq i \leq i_0\,,& p \mid a_i \\
		\forall j, \, 0 \leq j \leq j_0\,,& p \mid b_j
	\end{eqnarray*}
	On a donc que le coefficient de degré $i_0 + j_0$ de $PQ$ est:
	\begin{eqnarray*}
		PQ_{i_0+j_0}&=& \sum\limits^{i_0 + j_0}_{k=0} a_k b_{i_0 + j_0 - k} \\
		&=& a_{i_0}b_{j_0} + \text{un multiple de } p
	\end{eqnarray*}
	Si $k \neq i_0$, soit  $k\leq i_0 -1$ soit $i_0+j_0 - \leq j_0 -1$ et donc $ p \mid a_kb_{i_0 + j_0 - k}$\\
	Donc le coefficient de degré $i_0+j_0$ de $PQ$ n'est pas divisible par $p$ ce qui contredit les hypothèses.
	Donc on a $P$ et $Q$ primitif $\implies$ $PQ$ primitif. \\
	Dans le cas général
	\begin{eqnarray*}
		c(PQ) &=& c\left(\frac{P}{c(P)}\frac{Q}{c(Q)}c(P)c(Q)\right)\\
		&=& c\left(\frac{P}{c(P))}\frac{Q}{c(Q)}\right)c(P)c(Q)\\
		&=& c(P)c(Q)
	\end{eqnarray*}
	car $\frac{P}{c(P)}$ est un polynôme primitif de $A[X]$ et
	$pgcd(ka, kb) = k pgcd(a,b)$ et donc $c(kP) = kc(P)$.
\end{proof}


\begin{definition}[Corps de fraction]
	Soit $A$ un anneau  commutatif intègre.
	On introduit
	$$E = \left\{ (a,b) \in A\times A \mid b \neq 0 \right\}$$
	On munit  $E$ de 2 lois internes:
	\begin{itemize}
		\item $\times : (a,b) \times (a', b') = (aa',bb')$.
		\item $+ : (a,b) + (a', b') = (ab' + ba',bb')$.
	\end{itemize}

	On définit une relation d'équivalence $~$:
	$$ (a,b) \backsim  (a',b') \iff ab' = a'b $$

	Alors $K/\backsim $ (les classes d'équivalence de $E$ sur $\backsim$) est un corps et $A$ se plonge dans $K$ avec:
	$$\phi : a \in A  \mapsto \overline{(a,1)}$$
\end{definition}

\begin{remarque}
	$A, K  = Frac(A)$. Le plongement $\phi: A \to K$ nous permet d'identifier $A$ avec $\phi(A)$ de sorte que $A \subset K$.
	Ainsi un polynôme $P \in A[X]$  peut être vu comme un polynôme dans $K[X]$.
\end{remarque}

\begin{example}
	\begin{itemize}
		\item $Frac(\mathbb(Z)) = \mathbb{Q}$
		\item $Frac(\mathbb(K[X])) = K(X)$
	\end{itemize}
\end{example}

\begin{example}
	$2X^2 + 2X +2$ n'est pas irréductible dans $\mathbb{Z}[X]$ mais il est irréductible dans $\mathbb{Q}[X]$ car $2 \in \mathbb{Q}^\times$.
\end{example}



\begin{theorem}[Clasification des irréductibles] \href{https://fr.wikipedia.org/wiki/Lemme_de_Gauss_(polyn%C3%B4mes)#Applications}{Lemme de Gauss}\\
	Soit $A$ un anneau factoriel de corps de fraction $K$.
	Alors les irréductibles de $A[X]$ sont de deux types:
	\begin{itemize}
		\item Les polynômes constants $P = p$,  $p$ irréductibles dans $A$.
		\item Les polynômes primitifs de $\deg \geq 1 $ qui sont irréductibles dans $K[X]$.
	\end{itemize}
\end{theorem}


\begin{proof} On va traiter en premier les polynômes constants et après le reste.
	\begin{itemize}
		\item
		      Comme $A[X]^\times = A^\times$
		      si $P$ est constant, i.e. $P = p \in A$,
		      alors

		      $$ P \text{ inversible } \iff p \text{ inversible dans } A $$
		\item Montrons les deux implications pour les polynômes non constants.
		      \begin{itemize}
			      \item
			            Si $P$ primitif avec $\deg \geq 1$ dans $A[X]$ et irréductible dans $K[X]$, on écrit
			            $P = QR$, avec $Q,R \in A[X]$.

			            On a que $c(P) = c(Q) c(R) \in A^\times$ donc $c(Q) \in A^\times$ et $c(R) \in A^\times$. \\
			            La relation $P = QR \in K[X]$ implique que $Q$ ou $R$ sont de degré 0 (car $P$ primitif). \\
			            Comme ils sont primitifs, $Q \text{ ou } R \in A^\times \implies Q \text{ ou } R  \in A[X]^\times$
			            donc $P$ est irréductible dans $A[X]$.
			      \item
			            Soit $P$ irréductible de $A[X]$ avec $\deg \geq 1$. \\
			            Alors $c(P)$ divise $P$ donc $c(P) \in A^\times \implies P$ primitif.\\
			            Montrons maintenant que $P$ est irréductible dans $K[X]$.\\
			            On écrit $P = QR$ avec $Q,R \in K[X]$. On choisit $a,b \in A$ tel que $aQ \in A[X]$ et  $bR \in A[X]$.
			            On a donc que
			            \begin{eqnarray*}
				            abP &=& aQbR \\
				            c(aQ)c(bR) &=& c(abP) \\
				            &=& ab
			            \end{eqnarray*}
			            $$P = \frac{aQ}{c(aQ)} \frac{bR}{c(bR)} $$
			            donc $P$ produit de deux éléments de $A[X]$, donc, comme $P$ irréductible dans
			            $A[X]$, on a : $$ \deg (Q)= 0 \ \text{ou}\  \deg(R) = 0 $$
		      \end{itemize}
	\end{itemize}
\end{proof}

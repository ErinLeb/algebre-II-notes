\documentclass{article}
\usepackage[utf8]{inputenc}
\usepackage[T1]{fontenc}

\usepackage{amsmath}
\usepackage{amssymb} 
\usepackage{amsthm}  
\usepackage{dsfont}
\usepackage{mathrsfs}
\usepackage{mathtools}

\usepackage{geometry}

\usepackage{hyperref}        

\usepackage[french]{babel}

\usepackage[shortlabels]{enumitem}

\usepackage{fancyhdr}

\fancypagestyle{toc}{%
\fancyhf{}%
\fancyhead[L]{\rightmark}%
\fancyhead[R]{\thepage}%
}

\pagestyle{toc}

\newcommand{\indep}{\perp\!\!\! \perp}

\newcommand{\N}{\mathbb{N}}
\newcommand{\Z}{\mathbb{Z}}
\newcommand{\Zn}[1]{\mathbb{Z}/#1\mathbb{Z}}
\newcommand{\Q}{\mathbb{Q}}
\newcommand{\QX}{\mathbb{Q}[X]}
\newcommand{\R}{\mathbb{R}}
\newcommand{\RX}{\mathbb{R}[X]}
\newcommand{\C}{\mathbb{C}}
\newcommand{\CX}{\mathbb{C}[X]}
\newcommand{\K}{\mathbb{K}}
\newcommand{\AX}{A[X]}
\newcommand{\F}{\mathbb{F}}

\theoremstyle{plain}
\newtheorem{theorem}{Théorème}[section]
\newtheorem{coro}[theorem]{Corollaire}
\newtheorem{lemma}{Lemme}[section]
\newtheorem{prop}{Proposition}[section]

\theoremstyle{definition} 
\newtheorem{definition}{Définition}[section]
\newtheorem{example}{Exemple}[subsection]
\newtheorem{exercice}{Exercice}[subsection]

\theoremstyle{plain}
\newtheorem{remarque}{Remarque}[subsection]


\begin{document}
\begin{titlepage}
	\newcommand{\HRule}{\rule{\linewidth}{0.5mm}}
	\center

	\HRule\\[0.4cm]

	\textsc{\Large Algèbre II}\\[0.5cm]
	\textsc{\large Un ensemble compréhensible de notes de cours}\\[0.5cm]

	\HRule\\[1.5cm]


	{\large\textit{Auteur}}\\
	Yago \textsc{Iglesias}


	\vfill\vfill\vfill

	{\large\today}

	\vfill

\end{titlepage}


\tableofcontents


\section{Introduction}

Ce document est un recueil de notes de cours sur l'algèbre niveau L3. Il est
basé sur les cours de M.~\textsc{Régis de la Bretèche} à l'Université Paris Cité, cependant toute 
erreur ou inexactitude est de ma responsabilité.
Si bien \textsc{Yago IGLESIAS} est l'auteur de ce document, il n'est pas
le seul contributeur. En effet, de nombreux étudiants ont participé à la
rédaction de ce document. Leurs noms sont disponibles dans la section
contributeurs du répertoire \href{https://github.com/Yag000/algebre-II-notes/graphs/contributors}{GitHub}.
Un remerciement particulier à \textsc{Gabin Dudillieu} et \textsc{Mathusan Selvakumar} pour leur
participation active à la rédaction de ce document.
\vspace{0.5cm}

Le document est structuré en 3 parties. La première partie est consacrée aux anneaux et aux polynômes.
La deuxième partie est consacrée aux corps et aux extensions de corps. Et la dernière partie porte sur
la réduction des endomorphismes.
\vspace{0.5cm}

Toute erreur ou remarque est la bienvenue. 
Sentez vous libres de contribuer à ce document par le biais de \href{https://github.com/Yag000/algebre-II-notes}{GitHub}, 
où vous pouvez trouver le code source de ce document et une version pdf à jour.
Si vous n'etes pas familiers avec \textit{Git} ou \LaTeX , vous pouvez toujours me contacter
par \href{mailto: yago.iglesias.vazquez@gmail.com}{mail}.



\section{Anneaux de Polynômes}

\subsection{Construction formelle}

\begin{definition}[Anneau de polynômes]

	Soit $A$ un anneau commutatif. On note $A^{\mathbb{N}}$ l'ensemble des suites presques nulles d'éléments de $A$.
	On muni $A^{\mathbb{N}}$ d'une structure d'anneau en posant:
	\begin{itemize}
		\item $(a_n)_{n \in \mathbb{N}} + (b_n)_{n \in \mathbb{N}} = (a_n + b_n)_{n \in \mathbb{N}}$
		\item $(a_n)_{n \in \mathbb{N}} \cdot (b_n)_{n \in \mathbb{N}} = (c_n)_{n \in \mathbb{N}}$ où $c_n = \sum_{k=0}^{n} a_k b_{n-k}$
	\end{itemize}

	On note $A[X]$ l'anneau commutatif (proposition a montrer si besoin)  $A^{\mathbb{N}}$.

\end{definition}

\begin{definition}[Anneau intègre]
	Un anneau $(A, +, \cdot)$ est intègre si pour tout $x, y \in A$,
	\begin{equation*}
		xy = 0 \implies x = 0 \quad \text{ ou } \quad y = 0
	\end{equation*}
\end{definition}

\begin{prop}
	Soit $A$ un anneau commutatif.
	Soient $P, Q \in A[X]$. Alors $\deg(PQ) \leq \deg(P) + \deg(Q)$.
	De plus, si $A$ est intègre, $\deg(PQ) = \deg(P) + \deg(Q)$.
\end{prop}

\begin{proof}
	Soient $P=a_n x^n+\ldots+a_0$ et $Q=b_m x^m+$ $\cdots+b_0$ avec $a_n \neq 0$ et $b_m \neq 0$.
	Ainsi, $\deg(P)=n$ et $\deg(Q)=m$. Le terme de plus haut degré dans $P Q$ vient de $a_n X^n \cdot b_m X^m=a_n \cdot b_m X^{n+m}$.
	Par conséquent, $\deg(P Q) \leq n+m=\deg(P)+\deg(Q)$.

	Si A est intègre, alors $a_n b_m \neq 0$ si $a_n \neq 0$ et $b_m \neq 0$. Ainsi:
	$$
		\deg(P Q)=n+m=\deg(P)+\deg(Q).
	$$
\end{proof}

\begin{coro}
	$A[X]$ intègre $\iff A$ intègre.
\end{coro}

\begin{proof}
	$\Rightarrow$ c'est immédiat, car un sous-anneau d'un anneau intègre est intègre.

	$\Leftarrow$ Si $A$ trivial, alors $A[X]$ intègre est évident.
	Soient $P, Q \in A[X]$ non nuls. on a donc $\deg(P) \geqslant 0$ et $\deg(Q) \geqslant 0$.
	Mais alors, on a :
	$$
		\deg(P Q)=\deg(P)+\deg(Q) \geqslant 0, \text{par la proposition précédente}
	$$

	Ce qui entraine que $P Q$ est non nul, d'où l'intégrité de $A[X]$.
\end{proof}

\begin{example}
	$\mathbb{F}_p=\mathbb{Z}/p\mathbb{Z}$ est intègre, donc $\mathbb{F}_p[X] = \mathbb{Z}/p\mathbb{Z}[X]$ est intègre
\end{example}

\subsection{Division euclidienne}

\begin{prop}[Division euclidienne]
	Soit $A$ un anneau commutatif. Soient $P, Q \in A[X]$ avec $Q \neq 0$ et
	$Q$ a un coefficient dominant inversible. Alors il existe un unique couple
	$(U, R) \in A[X] \times A[X]$ tel que:
	\begin{itemize}
		\item $P = UQ + R$
		\item $\deg(R) < \deg(Q)$
	\end{itemize}
\end{prop}

\begin{prop}
	Soit $K$ un corps. $K[X]$ est un anneau euclidien et donc principal.
\end{prop}

\begin{proof}
	On commence par noter que $K$ est intègre et donc $K[X]$ aussi.
	Soit $I$ un idéal non nul de $K[X]$. On note $\mathcal{P}$ l'ensemble des degrés des polynômes de $I$:
	\[\mathcal{P} = \{ \deg(P) \mid P \in I \}\]

	$\mathcal{P}$ est non vide et minoré par $0$. Donc il existe $d \in \mathcal{P}$ tel que $d$ est minimal.
	On note $Q$ un polynôme de $I$ de degré $d$. Soit $P \in I$. On réalise la division euclidienne de $P$ par $Q$:
	\[ \exists (U, R) \in K[X] \times K[X] \mid P = UQ + R \text{ et } \deg(R) < \deg(Q) \]
	On a que $R = P - UQ \in I$  car $I$ est un idéal. Comme $\deg(R) < d$ et par définition $d$ est minimal pour
	tous les éléments non nuls, on a que $R = 0$. Donc $P = UQ$ et donc $I = <Q>$.
\end{proof}


\begin{example}[Exemple pathologique]
	$\mathbb{Z}[X]$ n'est pas principal. En effet, on peut prendre $I = <2, X>$.
\end{example}

\begin{theorem}[Propriété universelle de l'anneau de polynômes]\label{thm:prop_univ_anneau_poly}
	Soit $A$ un anneau commutatif et $B$ un anneau. On considère $f: A \to B$ un morphisme d'anneaux.
	Soit $b \in B$ tel que $f(a) = b$ pour tout $a \in A$. Alors il existe un unique morphisme d'anneaux
	$\tilde{f}: A[X] \to B$ tel que $\tilde{f}(X) = b$ et $\tilde{f}_{|A} = f$.
\end{theorem}

\begin{definition}[Evaluation]
	Soit $A$ un anneau commutatif et $a \in A$. On note $\phi_x: A[X] \to A$ le morphisme d'anneaux
	induit par l'automorphisme trivial de $A$ en utilisant la propriété universelle de l'anneau de polynômes \ref{thm:prop_univ_anneau_poly}.

	Ce morphisme correspond à l'évaluation en $x$ : $\phi_x(P) = P(x)$.

\end{definition}

\begin{definition}[Polynôme a plusieurs variables / à n indeterminés]
	Soit $A$ un anneau commutatif. On définit par récurrence sur $n \in \N^*$ l'anneau $A[X_1, \dots, X_n]$:
	\begin{itemize}
		\item $A[X_1] = A[X]$
		\item $A[X_1, \dots, X_n] = A[X_1, \dots, X_{n-1}][X_n]$
	\end{itemize}
\end{definition}


\begin{definition}[Anneau factoriel]
	Soit $A$ un anneau commutatif. On dit que $A$ est factoriel si:
	\begin{itemize}
		\item $A$ est intègre
		\item Tout élément non nul de $A$ est inversible ou est produit d'un nombre fini
		      d'éléments irréductibles ( $a = u p_1 \dots p_n$ avec $u$ inversible et $p_i$ irréductible)
		\item La décomposition est unique à l'ordre près et à l'association près.
	\end{itemize}
\end{definition}

\begin{prop}[Admis]
	Soit $A$ un anneau factoriel. Alors $A[X]$ est factoriel.
\end{prop}


\begin{prop}
	Soit $A$ un anneau intègre tel que tout élément de $A\setminus\{A^{\times}\}$ est produit d'un nombre fini d'éléments irréductibles, alors les assertions suivantes sont équivalentes:
	\begin{enumerate}
		\item $A$ est factoriel
		\item Si $p \in A$ est irréductible, alors l'idéal $<p>$ est premier
		\item Soient $a,b,c \in A\setminus\{0\}$ tels que $a \mid bc$ et $a$ et $b$ sont premiers entre eux. Alors $a \mid c$ (le lemme de Gauss).
	\end{enumerate}
\end{prop}

\begin{proof}

	3) $\implies$ 2):\\
	Soit $p \in A$ irréductible. On a que $<p> \neq A$ car $p$ est irréductible, donc pas inversible.
	Si $p$ divise $ab$ et ne divise pas $a$, alors $p$ et $a$ sont premiers entre eux car $p$ est irréductible.
	Donc tout diviseur commun de $p$ et de $c$ est soit inversible soit associé à $p$. Donc $p$ divise $c$.
	Donc d'après 3), $<p>$ est premier.

	\[ ab \in <p>\quad  et \quad a \notin <b> \implies b \in <p> \implies <p> \text{ premier} \]

\end{proof}

\begin{proof}

	2) $\implies$ 1):\\
	Soit $\mathcal{P}$ un système de représentants des irréductibles.
	\begin{equation*}
		u \cdot \prod_{p \in \mathcal{P}} p^{n_q} (\text{divisible par } q^{n_q} \in <q> ) = v \cdot \prod_{p \in \mathcal{P}} p^{m_q} (\text{divisible par } q^{m_q} \in <q>)
	\end{equation*}
	\noindent

	S'il existe $q \in \mathcal{P}$ tel que $m_q > n_q$, alors $q$ divise $u \cdot \prod\limits_{p \ne q} p^{n_q}$ ($\in <q>$), ce qui n'est pas possible par 2).

\end{proof}

\begin{proof}

	1) $\implies$ 3):\\
	$A$ est factoriel. Si $a$  divise $x$, on écrit $a$, $b$, $c$ sous la forme $u \cdot \prod\limits_{p \in \mathcal{P}} p^{v_{p(x)}}$.
	On a alors $\forall p \in \mathcal{P}$, $v_p(a) \leqslant v_p(b) \leqslant v_p(c)$ car $a$ divise $bc$.
	Si $v_p(a) \geqslant b$, alors $v_p(b) = 0$.
	Pour $H_p \in \mathcal{P}$, $v_p(a) \leqslant v_p(c)$, donc $a$ divise $c$, qui vérifie 3).

\end{proof}


\begin{definition}[pgcd]
	Le plus grand commun diviseur est défini ainsi:
	$$d = pgcd (a,b),\  \text{tout diviseur de} \ a  \ \text{et de} \ b \ \text{divise} \ d$$
	et $d$ divise $a$ et $b$
\end{definition}

\begin{prop}
	Si $A$ est un anneau factoriel, alors deux éléments non nuls de $A$ admettent un $pgcd$ défini à un facteur inversible près.
\end{prop}

\begin{proof}
	Soit $\mathcal{P}$ un système de représentants des irréductibles de $A$.
	On écrit
	$$ a = u \prod\limits_{p \in \mathcal{P} } p^{n_p} \ \text{où}\  u \in A^\times $$
	$$ b = v \prod \limits_{p \in \mathcal{P} } p^{m_p} \ \text{où}\  v \in A^\times $$
	$$ pgcd(a,b)= \prod \limits_{p \in \mathcal{P}} p^{\min(m_p, m_p)}\  \text{où} \  u \in A^\times $$
	à facteur $\omega \in A^\times$ près
\end{proof}

\begin{example}
	$$A  = \mathbb{Z}, \quad  pgcd(-6,2) = 2\ \text{ou}\ -2$$
\end{example}

\begin{theorem}[admis]
	$A$  principal $\implies A$ factoriel
\end{theorem}

\begin{prop}
	Dans un anneau principal on écrit
	$$ <a,b> \ = \  <d>, \quad \text{où}\quad d = pgcd (a,b) $$
\end{prop}

\begin{definition}
	Soit $A$ un anneau factoriel, et $P \in A[X]$, le contenu (notée $c(P)$) d'un polynôme $P$ est le
	$pgcd$ de ses coefficients non nuls.
	$P$ est dit primitif si $c(P)=1$ ( ou $c(P) \in A^\times$)
\end{definition}


\begin{example}
	$$A  = \mathbb{Z}, \quad  c(3X + 2) = 1$$
	$$A  = \mathbb{Z}, \quad  c(14X^2 + 24X + 2) = 2$$
\end{example}

\begin{remarque}
	Si $K$ est un corps, il y a un seul idéal non nul, qui est $K$ et donc
	tous les $pgcd$ valent $1$.
\end{remarque}

\begin{lemma}[Lemme de Gauss]
	Pour tout $P,Q \in A[X]$ on a
	$$ c(PQ) = c(P)c(Q)$$
	à facteur inversible près.
\end{lemma}

\begin{proof}
	Commençons par montrer que $P$ et $Q$ primitifs implique $PQ$ primitif. \\
	Sinon, il existe un irréductible $p \in A $ tel que $p$ divise tous les coefficients de $PQ$. \\

	Supposons que $P$ et $Q$ sont primitifs. On pose $P = \sum a_iX^i$ et  $Q = \sum b_jX^j$
	On a que $$D = \{i \,|\, p\  \text{ne divise pas}\  a_i \}$$ n'est pas vide, car si $D$ est vide alors $ \forall\  i,\, p \,|\, a_i \implies p | c(P)$.
	On note $i_0$ (resp. $j_0$) l'indice minimal tel que $a_{i_0}$ (resp. $b_{j_0}$) ne soit pas divisible par $p$ et :
	\begin{eqnarray*}
		\forall i, \, 0 \leq i \leq i_0\,,& p | a_i \\
		\forall j, \, 0 \leq j \leq j_0\,,& p | b_j
	\end{eqnarray*}
	On a donc que le coefficient de degré $i_0 + j_0$ de $PQ$ est:
	\begin{eqnarray*}
		PQ_{i_0+j_0}&=& \sum\limits^{i_0 + j_0}_{k=0} a_k b_{i_0 + j_0 - k} \\
		&=& a_{i_0}b_{j_0} + \text{un multiple de } p
	\end{eqnarray*}
	Si $k \neq i_0$, soit  $k\leq i_0 -1$ soit $i_0+j_o - \leq j_0 -1$ et donc $ p | a_kb_{i_0 + j_0 - k}$\\
	TODO: AJOUTER EXPLICATION EXTRA (en haut a gauche du tableau) \\

	Donc le coefficient de degré $i_0+j_0$ de $PQ$ n'est pas divisible par $p$ ce qui contredit les hypothèses.
	Donc on a $P$ et $Q$ primitif $\implies$ $PQ$ primitif. \\
	Dans le cas géneral
	\begin{eqnarray*}
		c(PQ) &=& c\left(\frac{P}{c(P)}\frac{Q}{c(Q)}c(P)c(Q)\right)\\
		&=& c\left(\frac{P}{c(P))}\frac{Q}{c(Q)}\right)c(P)c(Q)\\
		&=& c(P)c(Q)
	\end{eqnarray*}
	car $\frac{P}{c(P)}$ est un polynôme primitif de $A[X]$ et
	$pgcd(ka, kb) = k pgcd(a,b)$ et donc $c(kP) = kc(P)$.
\end{proof}


\begin{definition}[Corps de fraction]
	Soit $A$ un anneau  commutatif intègre.
	On introduit
	$$E = \left\{ (a,b) \in A\times A | b \neq 0 \right\}$$
	On munit  $E$ de 2 lois internes:
	\begin{itemize}
		\item $\times : (a,b) \times (a', b') = (aa',bb')$.
		\item $+ : (a,b) + (a', b') = (ab' + ba',bb')$.
	\end{itemize}

	On définit une relation d'équivalence $~$:
	$$ (a,b) \backsim  (a',b') \iff ab' = a'b $$

	Alors $K/\backsim $ (les classes d'équivalence de $E$ sur $\backsim$) est un corps et $A$ se plonge dans $K$ avec:
	$$\phi : a \in A  \mapsto \overline{(a,1)}$$
\end{definition}

\begin{remarque}
	$A, K  = Frac(A)$ Le plogement $\phi: A \to K$ nous permet identifier $A$ avec $\phi(A)$ de sorte que $A \subset K$.
	Ainsi un polynôme $P \in A[X]$  peut être vu comme un polynôme dans $K[X]$.
\end{remarque}

\begin{example}
	\begin{itemize}
		\item $Frac(\mathbb(Z)) = \mathbb{Q}$
		\item $Frac(\mathbb(K[X])) = K(X)$
	\end{itemize}
\end{example}

\begin{example}
	$2X^2 + 2X +2$ n'est pas irréductible dans $\mathbb{Z}[X]$ mais il est irréductible dans $\mathbb{Q}[X]$ car $2 \in \mathbb{Q}^\times$.
\end{example}



\begin{theorem}[Clasification des irréductibles] \href{https://fr.wikipedia.org/wiki/Lemme_de_Gauss_(polyn%C3%B4mes)#Applications}{Lemme de Gauss}\\
	Soit $A$ un anneau factoriel de corps de fraction $K$.
	Alors le irréductibles de $A[X]$ sont de deux types:
	\begin{itemize}
		\item Le polynômes constants $P = p$,  $p$ irréductible dans $A$.
		\item Le polynômes primitifs de $\deg \geq 1 $ qui sont irréductibles dans $K[X]$.
	\end{itemize}
\end{theorem}


\begin{proof} On va traiter en premier les polynômes constants et après le reste.
	\begin{itemize}
		\item
		      Comme $A[X]^\times = A^\times$
		      si $P$ est constant, i.e. $P = p \in A$,
		      alors

		      $$ P \text{ inversible } \iff p \text{ inversible dans } A $$
		\item Montrons les deux implications pour les polynômes non constants.
		      \begin{itemize}
			      \item
			            Si $P$ primitif avec $\deg \geq 1$ dans $A[X]$ et irréductible dans $K[X]$, on écrit
			            $P = QR$, avec $Q,R \in A[X]$.

			            On a que $c(P) = c(Q) c(R) \in A^\times$ donc $c(Q) \in A^\times$ et $c(R) \in A^\times$. \\
			            La relation $P = QR \in K[X]$ implique que $Q$ et $R$ sont de degré 0 (car $P$ primitif). \\
			            Comme ils sont primitifs, $Q \text{ ou } R \in A^\times \implies Q \text{ ou } R  \in A[X]^\times$
			            donc $P$ est irréductible dans $A[X]$.
			      \item
			            Soit $P$ irréductible de $A[X]$ avec $\deg \geq 1$. \\
			            Alors $c(P)$ divise $P$ donc $c(P) \in A^\times \implies P$ primitif.
			            Montrons que $P$ est irréductible dans $K[X]$.
			            On écrit $P = QR $ avec $Q,R \in K[X]$ On choisit $a,b \in A$ tel que $aQ \in A[X]$ et  $bR \in A[X]$.
			            On a donc que
			            \begin{eqnarray*}
				            abP &=& aQbR \\
				            c(aQ)c(bR) &=& c(abP) \\
				            &=& ab
			            \end{eqnarray*}
			            $$P = \frac{aQ}{c(aQ)} \frac{bR}{c(bR)} $$
			            donc $P$ produit de deux éléments de $A[X]$, donc comme P irréductible dans
			            $A[X]$ on a : $$ \deg (Q)= 0 \ \text{ou}\  \deg(R) = 0 $$
		      \end{itemize}
	\end{itemize}
\end{proof}

\subsection{Critères d'irréductibilité dans $K[X]$}

\begin{prop}
	Un polynôme de degré 1 dans $K[X]$ est irréductible dans $K[X]$.
\end{prop}

\begin{proof}
	Si $P = QR$, on a $ 1 = \deg(P) = \deg(Q) + \deg(R)$, donc $Q$ ou $R$ est de dégré 0, donc c'est un inversible.
\end{proof}

\begin{theorem}[d'Alambert-Gauss]
	Les irréductibles de $\CX$ sont les polynômes de degré 1.
\end{theorem}

\begin{proof}
	Tout polynôme $P\in \CX$ de degré $\geq 1$ admet $\alpha \in \C$ racine et donc $X-\alpha$ divise $P$.
\end{proof}

\begin{prop}
	Soit $P \in \mathbb{R}[X]$ irréductible alors:
	\begin{itemize}
		\item Soit $\deg P = 1$.
		\item Soit $\deg P = 2$ et $P$ n'admet pas de racines dans $\mathbb{R}$.
	\end{itemize}
\end{prop}

\begin{proof}
	$P(\alpha) = 0 \implies P(\bar{\alpha}) = 0$
	sur $\C \setminus \R, \, (X- \alpha) (X- \bar{\alpha}) = X ^2 - (\alpha + \bar{\alpha})X +
		\alpha\bar{\alpha} \in \RX$
	et divise $P$.
\end{proof}

\begin{prop}
	Soit $P \in K[X]$ de degré 2 ou 3,
	$P$ est irréductible dans $K[X]$ $\iff$ $P$ n'admet pas de racines dans $K$.
\end{prop}

\begin{example}[contre-exemple]
	$K = \R, P = (X^2 + 1 )^2$ n'a pas de racines dans $\R$, mais il n'est pas irréductible $(\deg P = 4 > 3)$.
\end{example}

\begin{example}
	$k = \Q, \,p = (x^2 + 2 )$ n'a pas de racines, donc irréductible dans $\QX$, mais il a des racines sur $\RX$, donc pas irrédutible sur $\RX$.
\end{example}

\begin{theorem}[Critère d'Eisenstein]
	Soit $A$ un anneau factoriel, $P$ un polynôme  de $\AX$ de degré $\geq 1$, $p$ un irréductible de $A$.
	On écrit $P = \sum\limits_{i=0}^n a_iX^i$ avec $a_i \neq 0$. Si on a les trois propriétés suivantes:
	\begin{itemize}
		\item $p$ ne divise pas $a_n$
		\item $p$ divise pas $a_k \ \forall \ k < n$
		\item $p^2$ ne divise pas $a_0$
	\end{itemize}
	Alors $P$ est irréductible dans $K[X]$(avec $K$ corps des fractions).
\end{theorem}

\begin{example}
	$P(X) = X^5 + 2X^4 + 2024X + 6\in \QX$. On a que $P$ est 2-eisenstein et donc irréductible.
\end{example}

\begin{coro}
	Si de plus $P$ est primitif de $\AX$, alors $P$ est irréductible dans $A[X]$.
\end{coro}


\begin{proof}
    Quitte à diviser par $c(P)$, on peut supposer $P$ primitif et de degré $\geq 2$.
	Si $P$ n'est pas irréductible, il s'écrit $P=RQ$, avec $R, Q \in A[X]$ de degré $>0$.
	On écrit $$Q = b_sX^s + \cdots + b_0$$ et $$R = c_rX^r + \cdots + c_0$$
	Soit $B = A/<p>$ intègre et on a $A[X]/pA[X] \simeq B[X]$.
	Dans $B[X]$, on a $\bar{P} = \bar{R}\bar{Q}$,
	Or d'après les hypothèses on a $\bar{P} = \bar{a}_0X^n$ et $\bar{a_n} \neq 0 $ dans $B$.
	Donc $\bar{b_s} \neq 0$ et $\bar{c_r} \neq 0$ et $\bar{b_s}\bar{c_r} = \bar{a_n}$ dans $B$,
	et $\bar{Q}$ et $\bar{R}$ sont de degré $>0$ et $\bar{Q}\bar{R} = \bar{a_n}X^n$ dans $B[X]$.

	On voit la relation $\bar{a_n}X^n= \bar{Q}\bar{R}$ dans $B[X]$ qui est principal et donc factoriel.


    $\bar{Q}$ et $\bar{R}$ sont des polynomes irréductibles dans $(Frac B)[X]$. Donc en particulier
	$p$ divise $b_0$ et $c_0$, donc $p^2$ divise $b_0c_0 = a_0$, ce qui est absurde.
\end{proof}


\begin{example}
	$A = \Z$, $p$ premier,\\
	$Q \in \Z[X] \implies \bar{Q} \in \Zn{p}[X]$.\\
	$\bar{Q}$ est défini par la classe $mod\  p$ de ses coefficients.
	$$X \mid \bar{Q} \text{ dans } \Zn{p}[X] \iff p \mid Q(0) \text{ dans } \Z$$
	\begin{eqnarray*}
		\Z[X] &\to& \Zn{p}[X] \\
		\sum a_iX^i &\mapsto& \sum \bar{a_i}X^i
	\end{eqnarray*}
\end{example}


\begin{example}
	$\Phi_p(x) = x^{p-1} + \cdots + x + 1  = \frac{x^p - 1}{x-1} \in \Z[X]$.
	On applique le critère d'Eisenstein à $\Phi_p(X+1)$.
	$$ \Phi_p(X+1) = \frac{(X+1)^p - 1}{X} = \sum_{k=1}^p \binom{p}{k} X^{k-1} = \sum_{k=0}^{p-1} \binom{p}{k+1} X^k$$
	\begin{itemize}
		\item Le coéfficient dominant de $X^{p-1}$ est $\binom{p}{p} = 1$ et $p$ ne divise pas 1.
		\item Pour tout $k < p-1$, le coefficient de $X^k$ est $\binom{p}{k+1} = \frac{p!}{(k+1)!(p-k-1)!} = p \underbrace{\frac{(p-1)!}{(k+1)!(p-k-1)!}}_{\in A}$.
		      $p$ divise $p! = k!(p-k)!\binom{p}{k}$ et $p$ premier à $k!(p-k)!$, donc $p$ divise $\binom{p}{k}$.
		\item $\binom{p}{1} = p$ et $p^2$ ne divise pas $p$. donc
	\end{itemize}
	Donc $\Phi_p(X+1)$ est irréductible dans $\Q[X]$ et donc $\Phi_p(X)$ est irréductible dans $\Q[X]$.
\end{example}


\begin{prop}
	Soi $P \in \Z[X]$ primitif de coefficient dominant non multiple de $p$, où $p$ est un premier.
	Si $\bar{P}$ est irréductible dans  $\Zn{p}[X]$, alors $P$ est irréductible dans $\Z[X]$.
\end{prop}

\begin{proof}
    $P$ est primitif, donc si P est est non irrédutible dans $\Z[X]$ alors il existe $Q, R \in \Z[X]$ non constants, tels que $P = QR$.
	et donc $\bar{P} = \bar{Q}\bar{R}$ dans $\Zn{p}[X]$.
	De plus, on a:
	\begin{itemize}
		\item $\deg Q = \deg \bar{Q}$
		\item $\deg R = \deg \bar{R}$

	\end{itemize}
	car leurs coefficients dominants ne divisent pas $p$.
	Donc $\bar{P} = \bar{Q}\bar{R} \implies \bar{P}$ n'est pas irréductible dans $\Zn{p}[X]$ ce qui contredit l'hypothèse.
\end{proof}

\subsection{Fonctions polynomiales}

\begin{definition}[Fonction polynomiale]
	Soit $P \in A[X]$, $P = \sum\limits_{i=0}^n a_iX^i$.\\
	On appelle fonction polynomiale associée à $P$ la fonction $\Pt:: A \to A$ définie par:
	$$\Pt(x) = \sum\limits_{i=0}^n a_ix^i$$
\end{definition}

\begin{example}
	$A = \Zn{p}$, $P = X^p - X$, avec $p$ premier.
	Alors
	$$\Pt(x) = x^p - x = 0 \href{https://fr.wikipedia.org/wiki/Petit_th%C3%A9or%C3%A8me_de_Fermat}{\text{ (petit théoreme de Fermat)}}$$
	Donc $\Pt$ est nulle sur $\Zn{p}$ mais $P$ n'est pas nul dans $\Zn{p}[X]$.
\end{example}

\begin{prop}
	Soient $P,Q$ deux polynômes de $A[X]$.
	\begin{itemize}
		\item $\Pt + \Qt = \widetilde{P+Q}$
		\item $\Pt \cdot \Qt = \widetilde{P \cdot Q}$
	\end{itemize}
	Et l'application $P \mapsto \Pt$ est un morphisme d'anneaux de $A[X]$ dans l'anneau des applications polynomiales.
	(En général il n'est pas injectif: voir exemple ci-dessus).
\end{prop}


\begin{definition}[Polynôme composé]
	Soit $P = \sum\limits_{i=0}^n a_iX^i \in A[X]$ un polynôme de $A[X]$ et $Q \in A[X]$. \\
	On appelle le polynôme composé de $P$ par $Q$ le polynôme
	$$P(Q) = a_nQ^n + a_{n-1}Q^{n-1} + \dots + a_0$$
	(On remplace l'indéterminé $X$ par le polynôme $Q$).
\end{definition}

\begin{prop}[Admis]
	$\widetilde{P(Q)} = \Pt \circ \Qt$
\end{prop}

\noindent
Par soucis de simplification on va noter $P(a)$ au lieu de $\Pt(a)$.

\begin{definition}[Racine d'un polynôme]
	Soit $P \in A[X]$.\\
	On dit que $a \in A$ est une racine de $P$ si et seulement si $P(a) = 0$.
\end{definition}

\begin{prop}
	$$ P(a) = 0 \iff (X-a) \mid P \text{ dans } A[X]$$
\end{prop}

\begin{proof}
	\begin{itemize}
		\item Si $P = (X-a)Q$, avec $Q \in A[X]$, alors $P(a) = 0$.
		\item Réciproquement, si $P(a) = 0$
		      \begin{eqnarray*}
			      P(X) &=& \sum_{i=0}^n b_iX^i \\
			      &=& \sum_{i=0}^n b_i(X)^i  - P(a), \ \text{car}  \ P(a) = 0 \\
			      &=& \sum_{i=0}^n b_i(X^i-a^i)
		      \end{eqnarray*}
		      On a que $X^i - a^i = (X-a)(X^{i-1} + X^{i-2}a + \dots + Xa^{i-2} + a^{i-1})$. %TODO: Add explanation\\
		      Donc $X-a$ divise $P$.
	\end{itemize}
\end{proof}

\begin{definition}[Multiplicité d'une racine]
	On dit que $a \in A$ est une racine de $P$ de multiplicité $m$ si et seulement si $(X-a)^m \mid P$ et $(X-a)^{m+1} \nmid P$.
\end{definition}

\begin{prop}
	Soit $A$ est un anneau intègre, alors $P \in A[X]$ admet au plus $\deg(P)$ racines distinctes dans $A$.\\
	Si $A = \C$ l'inégalité est une égalité.
\end{prop}

\begin{proof}
	On montre par récurrence sur $n = \deg(P)$ que $\sum\limits_{a \in A} m_p(a) \leq n$, ou $m_p(a)$ est la multiplicité de $a$ dans $P$.

	\begin{itemize}
		\item Initialisation: $n = 0$, $P = a_0 \in A$, $a_0 \neq 0$.
		      Alors $P$ n'a pas de racines dans $A$.
		\item Hérédité: Supposons que $\sum\limits_{a \in A} m_p(a) \leq n$. Soit $P \in A[X]$ de degré $n+1$.
		      \begin{itemize}
			      \item Premier cas, $P$ n'a pas de racines dans $A$.
			            Alors $\sum\limits_{a \in A} m_p(a) = 0 \leq n+1$.
			      \item Deuxième cas, il existe $b \in A$ tel que $P(b) = 0$.
			            Alors il existe $Q \in A[X]$ tel que $P = (X-b)Q$
			            et $\deg(Q) = n$. De plus
			            $$ m_P(a) = \left\{ \begin{array}{ll}
					            m_Q(a) + 1 & \text{si } a = b \\
					            m_Q(a)     & \text{sinon}
				            \end{array} \right. $$
			            et donc $\sum\limits_{a \in A} m_P(a) = \sum\limits_{a \in A} m_Q(a) + 1 \leq n+1$.
		      \end{itemize}

		      On a montré donc le résultat par récurrence.
	\end{itemize}
\end{proof}

\begin{coro}
	Si $A$ est intègre et possède un nombre infini d'éléments, alors les polynômes de $A[X]$ sont entièrement déterminés par leur fonctions polynomiales associées.
\end{coro}

\begin{definition}[Dérivée d'un polynôme]
	Soit $P \in A[X]$, $P = \sum\limits_{i=0}^n a_iX^i$. \\
	On appelle dérivée de $P$ le polynôme $P' = \sum\limits_{i=1}^n ia_iX^{i-1}$.\\
	Et on définit par récurrence $P^{(k)} = (P^{(k-1)})'$ et $P^{(0)} = P$.
\end{definition}

\begin{prop}
	On a la relation suivante:
	$$ (PQ)' = P'Q + PQ' $$
\end{prop}

\begin{lemma}
	Soit $A$ un anneau intègre, $P \in A[X]$ et $a \in A$.
	Alors:
	$$ m_p(a) = n \implies P^{(k)}(a) = 0 , \, \forall\, 0 \leq k \leq n-1$$
\end{lemma}

\begin{proof}
	On démontre le résultat par récurrence sur $n$, la multiplicité de $a$ dans $P$.
	\begin{itemize}
		\item Initialisation: Pour $n = 1$, $X-a \mid P$ et donc $P(a) = 0$
		\item Posons $n$ tel que c'est vrai pour tout polynôme de multiplicité $n$.
		      Soit $P$ tel que $m_p(a) = n+1$. Il existe $Q \in A[X]$ tel que $P = (X-a)^{n+1}Q$.
		      $$P' = (n+1)(X-a)^nQ + (X-a)^{n+1}Q' = (X-a)^n((n+1)Q + (X-a)Q')$$
		      Comme $Q(a) \neq 0$, on applique l'hypothèse de récurrence à $P$ et on obtient que
		      $$ \forall k \leq n-1 \quad (P')^{(k)}(a) = P^{(k+1)}(a) = 0 $$
	\end{itemize}
\end{proof}

\begin{prop}[Admis]
	Soit $A$ un anneau de caractéristique $0$, alors on a une formule de Taylor exacte pour les polynômes.
	$$\forall a \in A, \  P(X) = \sum\limits_{k=0}^n \frac{P^{(k)}(a)}{k!}(X-a)^k $$
	pour tout $n \geq \deg(P)$.
\end{prop}



\subsection{Clasification des ideaux premiers de $A[X]$, avec $A$ anneau principal}

\begin{theorem}
	Soit $A$ un anneau principal. Alors les idéaux premiers non nuls de $A[X]$ sont les idéaux de la liste suivante:
	\begin{itemize}
		\item $\goatp_{\pi} = <\pi>$ où $\pi$ est un élément irréductible de $A$. %TODO: Add { saying that they are the irreducible elements of A[X]}
		\item $\goatp_f = <f>$, où $f \in A[X]$ est un polynôme irréductible de degré $\geq 1$.
		\item $\goatm_{\pi,f} = <\pi,f>$, où $\pi$ est un élément irréductible de $A$ et $f \in A[X]$ unitaire, irréductible modulo $\pi$.
	\end{itemize}
	De plus
	\begin{itemize}
		\item $\goatp_{\pi}$ n'est pas maximal.
		\item Si $A$ possede une infinité d'éléments irréductibles deux à deux disjoints non associés, alors $p_f$ est premier mais pas maximal.
		\item $\goatm_{\pi,f}$ est maximal.
	\end{itemize}
\end{theorem}


\begin{rappel}[Irréductibilité]
	\begin{itemize}
		\item
		      Si $\pi$ est un élément irréductible de $A$, alors $A/<\pi>$ est un corps parce que dans un anneau
		      principal l'idéal engendré par un élément irréductible est maximal.\\
		\item Soit  $f \in A[X]$et $\bar{f}$ la classe de $f$ dans $A/<\pi>[X] = A[X]/<\pi>$.\\
		      alors $f$ irréductuble modulo $\pi$ signifie que $\bar{f}$ est irréductible dans $A/<\pi>[X]$.
	\end{itemize}
\end{rappel}

\begin{rappel}
	Si $f$ est un polynôme primitif de $A[X]$ de coéfficient dominant inversible modulo $\pi$, et $\bar{f}$ est irréductible dans $A/<\pi>[X]$, alors $f$ est irréductible dans $A[X]$.
\end{rappel}

\begin{proof}
	\begin{itemize}
		\item Si $\pi$ est un irréductible de $A$. Soit $g \in A[X]$ et on note $\bar{g} \in (A/<\pi>)[X]$ la classe de $g$ modulo $\pi$.
		      On rappelle
		      $$ A[X]/<\pi> \cong (A[X]/<\pi>A[X])$$
		      Comme $A$ est principal, $A/<\pi>$ est un corps et donc $(A/<\pi>)[X]$ est un anneau intègre (qui n'est pas un corps car les éléments
		      inversibles sont des polynômes constants). Donc $<\pi>$ est un idéal de $A[X]$ premier mais pas maximal.
		\item Soit $f \in A[X]$ de dégré $\geq 1$ et irréductible.$A$ principal $\implies A$ factoriel $\implies A[X]$ factoriel.
		      Tout élément irréductible de $A[X]$ engendre un idéal premier. \\
		      Si de plus $A$ possède une infinité d'éléments irréductibles deux à deux disjoints non associés, alors il exites $\pi \in A$ irréductinle de $A$
		      tel que $\pi$ ne divise pas le coefficient dominant de $f$.\\
		      Alors $<f>  \subset <\pi,f>$, et c'est une inclusion stricte.\\
		      En effet, si on avait $\pi \in <f>$ alors
		      $$\exists g \in A[X]\quad \pi = fg$$ ce qui n'est pas possible quand on regarde les dégrées \\
		      Montrons que $<\pi, f> \neq A[X]$.
		      Supposons que $<\pi, f> = A[X]$, autrent dit, il existe $g,h \in A[X]$ tels que $\pi g + fh = 1$.\\
		      $ \bar{f}\bar{g} = 1$ dans $(A/<\pi>)[X]$.\\
		      Comme $\bar{f}$ est inversible dans $(A/<\pi>)[X]$ et comme $A/<\pi>$ est intègre on a que $\deg \bar{f} = 0$.\\
		      Comme $\pi$ new divise pas le coefficient dominant de $f$, on a $\deg \bar{f} = \deg f = 0$, ce qui contredit $\deg f \geq 1$.\\
		      Donc $<\pi, f> \neq A[X]$ et $\goatp_f = <f>$ est un idéal premier mais pas maximal.
		\item  Soit $\pi$ irréductble de $A$, $f$ unitaire de $A[X]$ irréductible modulo $\pi$.\\
		      $$A[X] \to A/<\pi>[X] \to (A/<\pi>[X])/<f> $$ %TODO: Add phi
		      Le morphisme $\phi$ est surjectif de noyau $<\pi,f>$.
		      $$ A[X]/\goatm_{\pi,f} \cong (A/<\pi>[X]/<\bar{f}>)$$
		      Et donc $A/<\pi>[X]$ est principal.
		      $A$ est principal donc $<\pi>$ est maximal donc $A/<\pi>$ est un corps. Comme $A/<\pi>$ est un corps donc $(A/<\pi>)[X]$ est principal.\\
		      Comme $\bar{f}$ est un irréductible de $(A/<\pi>)[X]$, on a que $A/<\pi>[X]/<\bar{f}>$ est un corps.\\
		      Donc $A[X]/\goatm_{\pi,f}$ est un corps, donc $\goatm_{\pi,f}$ est maximal.
	\end{itemize}
	\vspace{0.25cm}
	\noindent Reciproquement, on choisit $\goatp$ un idéal non nul de $A[X]$ qui est premier. \\
	$\goatp \cap A$ est un idéal de A premier.
	Comme $A$ est principal, soit $\goatp \cap A = \{0\}$, soit il eciste $\pi$ irréductible de $A$ tel que $\goatp \cap A = <\pi>$.
	\begin{itemize}
		\item Supposons que $\goatp \cap A = <\pi>$.\\
		      On prend $\bar{\goatp}$ l'image de $\goatp$ dans $A/<\pi>[X]$ qui est principal.\\
		      $\bar{\goatp}$ est in idéal premier de l'a nneau principal $A/<\pi>[X]$.\\
		      \begin{itemize}
			      \item Soit $\bar{\goatp} = \{0\} \implies \goatp = <\pi>$.
			      \item Soit $\bar{\goatp}$ est engendré par un polynôme unitaie et irréductible de $A/<\pi>[X]$.
			            $$ \bar{\goatp} = <\bar{f}> $$
			            $f$ est un polynôme unitaire de $A[X]$ tel que $\bar{f}$est irréductible. \\
			            Donc il existe $g \in \goatp$ telle que $\bar{g} = \bar{f}$.
			            $$ \exists h \in A[X] \quad f = g + \pi h k$$
			            Comme $\pi \in \goatp$a, on a $f \in \goatp$.
			            Donc $\goatm_{\pi,f} \subset \goatp \subsetneq A$
			            donc $\goatm_{\pi,f} = \goatp$.
		      \end{itemize}
		\item Supposons que $\goatp \cap A = \{0\}$.\\
		      On choisit $f$ un élément non nul de $\goatp$ de degré minimal et par hypothèse, $\deg f \geq 1$.\\
		      On écrit $f = \alpha f_0$, avec $f_0 \in A[X]$ primitif et $\alpha \in A$.\\
		      Nous avons donc que $\alpha \in \goatp$ ou $f_0 \in \goatp$. \\
		      Or $\alpha \notin \goatp$ car $\goatp \cap A = \{0\}$, donc $f_0 \in \goatp$.\\
		      Suppososns $g \in \goatp$ non nu, alors $ \deg g \geq \deg f \geq 1$.
		      On écrit $g = hf + r$, avec $h \in K[X]$, ou $K$ est le corps des fractions de $A$ et $r \in K[X]$ de degré $< \deg f_0$.
		      (Division euclidienne dans $K[X]$).
		      On a $h \neq 0$ oar minimalité du degré de $f_0$. On aurait sinon
		      $g = r$ et $\deg r < \deg f_0$, ce qui est absurde.\\
		      On choisit $d in A$ tel que $dh$ et $df_0$ soient dans $A[X]$.\\
		      $$ dr = \underbrace{d}_{\in A} \underbrace{g}_{\in \goatp} - \underbrace{dh}_{\in A[X]} \underbrace{f_0}_{\in \goatp} \in \goatp$$
		      $$ dg =f_0hd \text{ et } c(dg) = c(f_0)c(dh) = c(dh) = dc(g)$$
		      donc $h \in A[X]$ car $g = f_0h$ et donc $\goatp = <f_0>$.
	\end{itemize}
\end{proof}



\subsection{Fractions rationelles}

\begin{definition}
	On écrit tous les éléments de $K(X)$ sous la forme $\frac{P}{Q}$ avec $P,Q \in K[X]$ et $Q \neq 0$.
\end{definition}

\begin{theorem}[Décomposition en éléments simples]
	Soit $K$ un corps. toute fraction rationelle $F = \frac{P}{Q} \in K(X)$ admet une décomposition comme somme
	d'éléments simples, c'est-à-dire comme la somme d'un polynôme $T \in K[X]$ (appelé partie entière de F) et de fractions
	$\frac{J}{H^k}$ où $J,H \in K[X]$, $H$ irréductible, $k \geq 1$ et $\deg J < \deg H$. \\

	De plus si $Q = H_1^{k_1} \cdots H_q^{k_q}$ et $P$ et $Q$ premiers entre eux, alors

	$$ F = \frac{P}{Q} = T + F_1 + \cdots + F_q$$
	$$\text{Où } F_i = \frac{J_{i,1}}{H_i} + \frac{J_{i,2}}{H_i^2} + \cdots + \frac{J_{i,n_i}}{{H_i}^{n_i}}$$
\end{theorem}






\end{document}
